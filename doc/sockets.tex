\ignore{
\documentstyle[11pt]{report}
\textwidth 13.7cm
\textheight 21.5cm
\newcommand{\myimp}{\verb+ :- +}
\newcommand{\ignore}[1]{}
\def\definitionname{Definition}

\makeindex
\begin{document}

}
\chapter{Sockets}
Sockets\index{socket} allow communication across computer networks.  Picat's \texttt{socket} module enables both connection-oriented and connectionless communication.  This module supports communication in the Internet domain\index{internet domain} and in the Unix domain\index{unix domain}.  All Picat programs that use sockets\index{socket} must import the \texttt{socket} module.

\section{\label{tcp}Connection-Oriented Communication}

Connection-oriented communication uses TCP\index{tcp}, the Transmission Control Protocol.  Before data is transmitted, a client and a server establish a full-duplex connection.  TCP\index{tcp} is reliable, meaning that it checks for errors during transmission, and that it sequences packets that arrive in the wrong order.  The following code skeleton shows how a server and a client communicate over TCP\index{tcp}.

\subsection*{Example}
\begin{verbatim}
import socket, process.

server =>
     FD = socket(inet, stream),
     Port = bind(FD, inet, inaddr_any, 0),
     listen(FD), % wait for connection
     do          % infinite loop
          Client = accept(FD),
          P = process.fork(),
          if P == 0 then     % child process
             echo(Client.client_fd)
          else               % parent process
             close(Client.client_fd)
          end
     while (true).

client(Address, Port) =>
     FD = socket(inet, stream),
     connect(FD, inet, Address, Port),
     hello_to_server(FD),
     close(FD). 

echo(Client) => % Server code to communicate with client
     % Begin communication
     send(Client, "Enter Input:"),
     Str = recv(Client), % wait for client input
     send(Client, Str),
     % End communication
     close(Client).

hello_to_server(FD) => % Client code to communicate with server
     % Begin communication
     Str = recv(FD), % wait for server's message
     println(Str)
     send(FD, "Hello"),
     Str := recv(FD),
     println(Str).
     % End communication
\end{verbatim}

The server performs the following steps:
\begin{enumerate}
\item The server creates a communication endpoint, using the \texttt{socket}\index{\texttt{socket/2}} function.
\item The server associates itself with a port number, using \texttt{bind}\index{\texttt{bind/4}}.
\item The server waits for connections, using the \texttt{listen}\index{\texttt{listen/1}} predicate.
\item When a connection request arrives, the server handles the connection, using the \texttt{accept}\index{\texttt{accept/1}} function. 
\item For connection-oriented communication, multiple messages can be exchanged.  Therefore, if the server is capable of receiving multiple requests from different clients, the server can fork a new process\index{process} for each request.  The new process\index{process} communicates with the client by using \texttt{send}\index{\texttt{send/2}} and \texttt{recv}\index{\texttt{recv/1}}, while the parent process\index{parent process} closes its copy of the client's file descriptor, and continues listening for other clients.  Otherwise, the server receives one request at a time, meaning that there is only one process\index{process}, which finishes communicating with one client before listening for another client.
\end{enumerate}

The client performs the following steps:
\begin{enumerate}
\item The client creates a communication endpoint, using the \texttt{socket}\index{\texttt{socket/2}} function.
\item The client tries to connect to the server, using the \texttt{connect}\index{\texttt{connect/4}} predicate.
\item After the server accepts the connection request, the client communicates with the server by using \texttt{send}\index{\texttt{send/2}} and \texttt{recv}\index{\texttt{recv/1}}.
\item When communication is complete, the client closes the socket's\index{socket} file descriptor.
\end{enumerate}

The following functions and predicates allow a server and a client to communicate over TCP\index{tcp}:

\begin{itemize}
\item \texttt{socket($Domain$, $Type$) = $FD$}\index{\texttt{socket/2}}: This returns a file descriptor for the communication endpoint.  The $Domain$ parameter can be \texttt{inet}, \texttt{inet6}, or \texttt{unix}.  For communication over the Internet domain\index{internet domain}, use either \texttt{inet} or \texttt{inet6}.  The domain \texttt{unix} allows communication over the Unix domain\index{unix domain}.  The domain \texttt{inet} uses IP\index{ip} version 4, and the domain \texttt{inet6} uses IP\index{ip} version 6.  The $Type$ parameter can be one of the following five atoms: \texttt{stream}, \texttt{dgram}, \texttt{raw}, \texttt{seqpacket}, or \texttt{rdm}.  The \texttt{stream} atom allows connection-oriented communication over TCP\index{tcp}.  The \texttt{dgram} atom allows connectionless communication over UDP\index{udp}.  The other atoms are used as follows: \texttt{raw} is used for checking communication paths, \texttt{seqpacket} allows reliable and bidirectional transmission of packets, and \texttt{rdm} is used for reliably-delivered messages.  
\item \texttt{tcp\_socket() = $FD$}\index{\texttt{tcp\_socket/0}}: This performs the same operation as \texttt{socket(inet, stream)}\index{\texttt{socket/2}}.  It creates a TCP\index{tcp} socket\index{socket} that uses IPv4\index{ip}.
\item \texttt{bind($FD$, $INet$, $Address$, $Port$)}\index{\texttt{bind/4}}: This associates a communication endpoint with a port number.  The $FD$ parameter is the file descriptor for the communication endpoint that the \texttt{socket}\index{\texttt{socket/2}} function returns.  In the Internet domain\index{internet domain}, the $INet$ parameter must be either \texttt{inet}, which specifies that IPv4\index{ip} is used, or \texttt{inet6}, which specifies that IPv6\index{ip} is used.  The $Address$ parameter is a string that specifies the IP\index{ip} address that will be used for the connections.  If IPv4\index{ip} is used, then $Address$ is an address string in dotted-decimal notation, such as ``127.0.0.1".  If IPv6\index{ip} is used, then $Address$ is an address string in hexadecimal notation, such as ``2001:DB8:0:0:8:800:200C:417A".  The $Address$ parameter can be the atom \texttt{inaddr\_any}, which indicates that the server will listen for connections on any of the network interfaces.  It can also be the atom \texttt{localhost}, which represents the address of the local machine.  The $Port$ parameter is an integer that specifies the port that should be used; set $Port$ to $0$ in order to allow the operating system to pick the port that should be used. 
\item \texttt{tcp\_bind($FD$, $Address$, $Port$)}\index{\texttt{tcp\_bind/3}}: This is the same as \texttt{bind($FD$, inet, $Address$, $Port$)}\index{\texttt{bind/4}}.   It uses IPv4\index{ip}.
\item \texttt{listen($FD$, $Backlog$)}\index{\texttt{listen/2}}: This is only used by the server.  Listening causes the server to wait for incoming connection requests.  The $FD$ parameter is the server's file descriptor, which the \texttt{socket}\index{\texttt{socket/2}} function returns.  The $Backlog$ parameter specifies the maximum number of pending client connection requests.
\item \texttt{listen($FD$)}\index{\texttt{listen/1}}: This is only used by the server.  This function uses a default backlog of $5$.
\item \texttt{accept($FD$) = $Client$}\index{\texttt{accept/1}}: This is only used by the server.  This function accepts incoming connections.  The $FD$ parameter is the server's file descriptor, which the \texttt{socket}\index{\texttt{socket/2}} function returns.  The \texttt{accept}\index{\texttt{accept/1}} function returns a map with the keys \texttt{client\_fd}, \texttt{client\_domain}, \texttt{client\_address}, and \texttt{client\_port}.  The key \texttt{client\_fd} stores the file descriptor that is used to communicate with the client.  The key \texttt{client\_domain} indicates the domain that the client is using to communicate, as specified in the \texttt{socket}\index{\texttt{socket/2}} function.  The key \texttt{client\_address} is a string that stores the client's IP\index{ip} address.  The key \texttt{client\_port} stores the port number that the client is using to receive data from the server.
\item \texttt{connect($FD$, $INet$, $Address$, $Port$)}\index{\texttt{connect/4}}: This is only used by the client.  It causes the client to send a connection request to the server.  The $FD$ parameter is the client's file descriptor, which the \texttt{socket}\index{\texttt{socket/2}} function returns.  In the Internet domain\index{internet domain}, the $INet$ parameter is either the atom \texttt{inet}, specifying that IPv4\index{ip} is used, or the atom \texttt{inet6}, specifying that IPv6\index{ip} is used.  The $Address$ parameter is a string that specifies the server's IP\index{ip} address, as described in the explanation of \texttt{bind}\index{\texttt{bind/4}}.  The $Address$ parameter can be the atom \texttt{localhost} in order to represent the address of the local machine.   The \texttt{connect}\index{\texttt{connect/4}} predicate requires a $Port$ parameter in the Internet domain\index{internet domain}, specifying the number of the port on which the server will receive the connection. 
\item \texttt{tcp\_connect($FD$, $Address$, $Port$)}\index{\texttt{tcp\_connect/3}}: This performs the same operation as \texttt{connect($FD$, inet, $Address$, $Port$)}\index{\texttt{connect/4}}.  It connects TCP\index{tcp} sockets\index{socket} that use IPv4\index{ip}.
\item \texttt{send($FD$, $Message$) = $NBytes$}\index{\texttt{send/2}}: This is used for reliable communication.  It blocks until it is able to send the message.  The $FD$ parameter is the file descriptor that the \texttt{socket}\index{\texttt{socket/2}} function returns.  The $Message$ parameter is a string that one endpoint is transmitting to the other endpoint.  The \texttt{send}\index{\texttt{send/2}} function returns the number of bytes that were sent.
\item \texttt{send($FD$, $Message$, $Flags$) = $NBytes$}\index{\texttt{send/3}}: The $Flags$ parameter is a list of options. The options can be \texttt{oob}, \texttt{dontroute}, \texttt{dontwait}, and \texttt{nosignal}.  The option \texttt{oob} is used to indicate out-of-band urgent data.  The \texttt{dontroute} option indicates that the data should not be sent over a router.  The \texttt{dontwait} option indicates that, instead of blocking, if \texttt{send}\index{\texttt{send/3}} cannot immediately transmit the message, then it should throw an error.  The \texttt{nosignal} option prevents a signal from being raised if a message is sent to a host that is not receiving.  Windows only supports the \texttt{oob} and \texttt{dontroute} options.
\item \texttt{recv($FD$) = $Message$}\index{\texttt{recv/1}}: This is used for reliable communication.  It blocks until data arrives.  The $FD$ parameter is the file descriptor that the \texttt{socket}\index{\texttt{socket/2}} function returns.  The \texttt{recv}\index{\texttt{recv/1}} function blocks until data is received, and returns the message that was received.  The \texttt{recv}\index{\texttt{recv/1}} function returns a message string.  
\item \texttt{recv($FD$, $Flags$) = $Message$}\index{\texttt{recv/2}}:  The $Flags$ parameter is a list of options.   The options can be \texttt{oob}, \texttt{peek}, \texttt{waitall(Length)}, and \texttt{dontwait}.  The option \texttt{oob} is used to receive out-of-band urgent data.  The \texttt{peek} atom indicates that the receiver should look at the message without removing it from the incoming message buffer.  The \texttt{waitall(Length)} option  indicates that the function should block until $Length$ bytes are received.  The option \texttt{dontwait} indicates that the function should return immediately, even if data is unavailable.
\item \texttt{close($FD$)}\index{\texttt{close/1}}:  This performs the same operation as the \texttt{close}\index{\texttt{close/1}} predicate in the \texttt{io} module.
\end{itemize}
Binding and connecting in the Unix domain\index{unix domain} will be discussed in Section \ref{unix_domain}.

The endpoints can also communicate by using the built-ins from the \texttt{io} module.  However, further management is required.  For example, the \texttt{fread} functions are non-blocking.  If the other communication endpoint did not yet send data, the \texttt{fread} functions would return \texttt{eof}\index{\texttt{eof}} immediately.

The following example shows how a client and server would send data to each other by using \texttt{io} built-ins.

\subsection*{Example}
\begin{verbatim}
import io.

echo(Client) =>
     % Begin communication
     io.fprintln(Client, "Enter Input:"),
     Str = io.fread_file(Client),
     while (Str == eof)
            Str := io.fread_file(Client)
     end,
     io.fprintln(Client, Str),
     % End communication
     close(Client).

hello_to_server(FD) =>
     % Begin communication
     Str = io.fread_file(FD),
     while (Str == eof)
            Str := io.fread_file(FD)
     end,          
     println(Str),
     io.fprintln(FD, "Hello"),
     Str = io.fread_file(FD),
     while (Str == eof)
            Str := io.fread_file(FD)
     end,          
     println(Str).
     % End communication     
\end{verbatim}

\section{Connectionless Communication}
There are times when connection-oriented communication has too much overhead.  The transmission reliability is not necessary, faster connectionless communication can be used.  Connectionless communication occurs over UDP\index{udp}, the User Datagram Protocol.  Unlike TCP\index{tcp}, UDP\index{udp} does not require connection establishment and termination.  UDP\index{udp} is unreliable, meaning that there can be errors during transmission, and data can arrive out of order.  UDP\index{udp} can be used with applications that only require a simple request and response, such as DNS or NTP, multicasting\index{multicast} communication, and real-time applications, such as video.  The following code skeleton shows how a server and a client communicate over UDP\index{udp}.

\subsection*{Example}
\begin{verbatim}
import socket.

server =>
     FD = socket(inet, dgram),
     Port = bind(FD, inet, inaddr_any, 0),
     while (true)                  % infinite loop
            Message = recvfrom(FD, inet), 
            sendto(FD, "Message Received", inet, 
                   Message.address, Message.port)
     end.

client(Address, Port) =>
     FD = socket(inet, dgram),
     MyPort = bind(FD, inet, inaddr_any, 0),
     sendto(FD, "Did you get this message?", 
            inet, Address, Port),
     Message = recvfrom(FD, inet),
     close(FD). 
\end{verbatim}

The server performs the following steps:
\begin{enumerate}
\item The server creates a communication endpoint, using the \texttt{socket}\index{\texttt{socket/2}} function.
\item The endpoint associates itself with a port number, using \texttt{bind}\index{\texttt{bind/4}}.
\item In an infinite loop, the following occurs:
\begin{enumerate}
\item The server uses \texttt{recvfrom}\index{\texttt{recvfrom/2}} to wait for an incoming message.
\item The server uses \texttt{sendto}\index{\texttt{sendto/5}} to respond to the client's message.
\end{enumerate}
\end{enumerate}

The client performs the following steps:
\begin{enumerate}
\item The client creates a communication endpoint, using the \texttt{socket}\index{\texttt{socket/2}} function.
\item The client endpoint associates itself with a port number, using \texttt{bind}\index{\texttt{bind/4}}.
\item The client uses \texttt{sendto}\index{\texttt{sendto/5}} to send a message to the server.
\item The client uses \texttt{recvfrom}\index{\texttt{recvfrom/2}} to wait for the server's response.
\item When communication is complete, the client closes the socket's\index{socket} file descriptor.
\end{enumerate}

Note that in the TCP\index{tcp} example, both the client and the server used the client's socket's\index{socket} file descriptor for \texttt{send}\index{\texttt{send/2}} and \texttt{recv}\index{\texttt{recv/1}}, while in the UDP\index{udp} example, each endpoint uses its own file descriptor for \texttt{sendto}\index{\texttt{sendto/5}} and \texttt{recvfrom}\index{\texttt{recvfrom/2}}.

Communication between a client and a server over UDP\index{udp} uses the following functions and predicates: 

\begin{itemize}
\item \texttt{socket($Domain$, $Type$) = $FD$}\index{\texttt{socket/2}}: For UDP\index{udp}, the socket's\index{socket} $Type$ parameter is \texttt{dgram}.
\item \texttt{udp\_socket() = $FD$}\index{\texttt{udp\_socket/0}}: This performs the same operation as \texttt{socket(inet, dgram)}\index{\texttt{socket/2}}.
\item \texttt{bind($FD$, $INet$, $Address$, $Port$)}\index{\texttt{bind/4}}: This associates a communication endpoint with a port number, as discussed in Section \ref{tcp}.  Note that for UDP\index{udp}, both the server and the client must call \texttt{bind}\index{\texttt{bind/4}}. 
\item \texttt{udp\_bind($FD$, $Address$, $Port$)}\index{\texttt{udp\_bind/3}}: This performs the same operation as \texttt{bind($FD$, inet, $Address$, $Port$)}\index{\texttt{bind/4}}.  It uses IPv4\index{ip}.
\item \texttt{sendto($FD$, $Message$, $Domain$, $Address$, $Port$) = $NBytes$}\index{\texttt{sendto/5}}: This is used for unreliable communication.  This function blocks until it is able to send the message.  It can use either IPv4\index{ip} or IPv6\index{ip}.  The $FD$ parameter is the file descriptor that the \texttt{socket}\index{\texttt{socket/2}} function returns.  Unlike the \texttt{send}\index{\texttt{send/2}} function, where the domain is known from \texttt{connect}\index{\texttt{connect/4}} and \texttt{accept}\index{\texttt{accept/1}}, the \texttt{sendto}\index{\texttt{sendto/5}} function must specify the domain.  The $Domain$ parameter can be either \texttt{inet} or \texttt{inet6}.  The $Message$ parameter is a string that one endpoint is transmitting to the other endpoint.  The $Address$ parameter is a string that stores the IP\index{ip} address of the other endpoint, as described in the explanation of \texttt{bind}\index{\texttt{bind/4}}.  The $Address$ parameter can be the atom \texttt{localhost} in order to represent the address of the current machine.  The $Port$ parameter stores the number of the port that the other endpoint is using to communicate.  The \texttt{sendto}\index{\texttt{sendto/5}} function returns the number of bytes that were sent.  See Section \ref{unix_domain} for \texttt{sendto}\index{\texttt{sendto/3}} in the Unix domain\index{unix domain}.
\item \texttt{sendto($FD$, $Message$, $Flags$, $Domain$, $Address$, $Port$) = $NBytes$}\index{\texttt{sendto/6}}: The $Flags$ parameter is a list of options. The options can be \texttt{oob}, \texttt{dontroute}, \texttt{dontwait}, and \texttt{nosignal}.  These options were discussed in the description of \texttt{send}\index{\texttt{send/3}} in Section \ref{tcp}.  Windows only supports \texttt{oob} and \texttt{dontroute}.
\item \texttt{recvfrom($FD$, $Domain$) = $Message$}\index{\texttt{recvfrom/2}}: This is used for unreliable communication.  It can use either IPv4\index{ip} or IPv6\index{ip}.  The $FD$ parameter is the file descriptor that the \texttt{socket}\index{\texttt{socket/2}} function returns.  Unlike the \texttt{recv}\index{\texttt{recv/2}} function, where the domain is known from \texttt{connect}\index{\texttt{connect/4}} and \texttt{accept}\index{\texttt{accept/1}}, the \texttt{recvfrom}\index{\texttt{recvfrom/2}} function must specify the domain.  The $Domain$ parameter can be \texttt{inet} or \texttt{inet6}.  In the Unix domain\index{unix domain}, it can also be \texttt{unix}.  The \texttt{recvfrom}\index{\texttt{recvfrom/2}} function blocks until data is received, and returns the message that was received.  Unlike the \texttt{recv}\index{\texttt{recv/2}} function, the \texttt{recvfrom}\index{\texttt{recvfrom/2}} function returns a map.  Since the communication is connectionless, information about the other communication endpoint is not known until a message is received.  Therefore, the message is stored in a map, which contains information about the sender.  In the Internet domain\index{internet domain}, the \texttt{recvfrom}\index{\texttt{recvfrom/2}} function returns a map with the keys \texttt{address}, \texttt{port}, and \texttt{message}.  The key \texttt{address} is a string that stores the IP\index{ip} address of the other endpoint.  The key \texttt{port} stores the number of the port that the other endpoint is using to communicate.  The key \texttt{message} contains the actual message string that was received.  See Section \ref{unix_domain} for \texttt{recvfrom}\index{\texttt{recvfrom/2}} in the Unix domain\index{unix domain}.
\item \texttt{recvfrom($FD$, $Flags$, $Domain$) = $Message$}\index{\texttt{recvfrom/3}}:  The $Flags$ parameter is a list of options.  The options can be \texttt{oob}, \texttt{peek}, \texttt{waitall(Length)}, and \texttt{dontwait}.  These options were discussed in the description of \texttt{recv}\index{\texttt{recv/2}} in Section \ref{tcp}.  Windows only supports \texttt{oob} and \texttt{peek}.
\item \texttt{close($FD$)}\index{\texttt{close/1}}
\end{itemize}

\section{Multicasting}
Multicasting\index{multicast} allows a group of receivers to receive a single message.  Multicasting\index{multicast} is implemented using UDP\index{udp}.  The following example shows how multicasting\index{multicast} communication can occur.

\subsection*{Example}
\begin{verbatim}
import socket.

sender(GroupAddress, GroupPort) =>
     FD = socket(inet, dgram),
     while (true)                  % infinite loop
            sendto(FD, "Group Message", inet, 
                   GroupAddress, GroupPort)
     end.

receiver(GroupAddress, GroupPort)  =>
     Messages = new_array(100),
     FD = socket(inet, dgram),
     MyPort = bind(FD, inet, inaddr_any, 0),
     joingroup(GroupAddress),
     foreach (I in 1 .. 100) % receive 100 messages
              Messages[I] = recvfrom(FD, inet)
     end,
     leavegroup(GroupAddress),
     close(FD). 
\end{verbatim}

A sender performs the following steps:
\begin{enumerate}
\item In order to send a message to a group, a sender creates a UDP\index{udp} communication endpoint, using the \texttt{socket}\index{\texttt{socket/2}} function.
\item The sender uses the \texttt{sendto}\index{\texttt{sendto/5}} function, specifying the group's IP\index{ip} address.  This will send a single message to every member of the group.
\end{enumerate}

Each receiver performs the following steps:
\begin{enumerate}
\item A receiver creates a UDP\index{udp} communication endpoint, using the \texttt{socket}\index{\texttt{socket/2}} function.
\item The receiver chooses a port through which to receive group messages, using the \texttt{bind}\index{\texttt{bind/4}} function.
\item The receiver requests to join the group, using the \texttt{joingroup}\index{\texttt{joingroup/1}} predicate.
\item The receiver uses the \texttt{recvfrom}\index{\texttt{recvfrom/2}} function to listen for incoming datagrams.
\item If a receiver would like to leave the group, it uses the \texttt{leavegroup}\index{\texttt{leavegroup/1}} predicate.  In the above example, the receiver leaves the group after receiving $100$ group messages.
\item After the receiver leaves the group, it closes the socket's\index{socket} file descriptor.
\end{enumerate}

Multicasting\index{multicast} uses the following functions and predicates:
\begin{itemize}
\item \texttt{socket($Domain$, $Type$) = $FD$}\index{\texttt{socket/2}} 
\item \texttt{udp\_socket() = $FD$}\index{\texttt{udp\_socket/0}}: This performs the same operation as \texttt{socket(inet, dgram)}\index{\texttt{socket/2}}. 
\item \texttt{bind($FD$, $INet$, $Address$, $Port$)}\index{\texttt{bind/4}}: The receiver associates a communication endpoint with a port number.  Note that the sender is not expecting a response, so the sender does not need to bind itself to a port. 
\item \texttt{udp\_bind($FD$, $Address$, $Port$)}\index{\texttt{udp\_bind/3}}: This performs the same operation as \texttt{bind($FD$, inet, $Address$, $Port$)}\index{\texttt{bind/4}}.  It uses IPv4\index{ip}.
\item \texttt{joingroup($GroupAddress$)}\index{\texttt{joingroup/1}}: This allows a receiver to request to join a group.  The $GroupAddress$ parameter is a string specifying the group's IP\index{ip} address in dotted-decimal notation (in IPv4\index{ip}) or hexadecimal notation (in IPv6\index{ip}).  The \texttt{joingroup}\index{\texttt{joingroup/1}} predicate performs the same operation as \texttt{setsockopt($FD$, ip, addmembership, $GroupAddress$)}\index{\texttt{setsockopt/4}}.  For a further discussion of \texttt{setsockopt}\index{\texttt{setsockopt/4}}, see Section \ref{socket_options}.
\item \texttt{leavegroup($GroupAddress$)}\index{\texttt{leavegroup/1}}: This allows a receiver to leave a group.  The $GroupAddress$ parameter is a string specifying the group's IP\index{ip} address.  The \texttt{leavegroup}\index{\texttt{leavegroup/1}} predicate performs the same operation as \texttt{setsockopt($FD$, ip, dropmembership, $GroupAddress$)}\index{\texttt{setsockopt/4}}.  For a further discussion of \texttt{setsockopt}\index{\texttt{setsockopt/4}}, see Section \ref{socket_options}.
\item \texttt{sendto($FD$, $Message$, $Domain$, $Address$, $Port$) = $NBytes$}\index{\texttt{sendto/5}}: This is used for unreliable communication.  This function blocks until it is able to send the message.  The \texttt{sendto}\index{\texttt{sendto/5}} function returns the number of bytes that were sent.  
\item \texttt{sendto($FD$, $Message$, $Flags$, $Domain$, $Address$, $Port$) = $NBytes$}\index{\texttt{sendto/6}}: The $Flags$ parameter is a list of options. The options can be \texttt{oob}, \texttt{dontroute}, \texttt{dontwait}, and \texttt{nosignal}.
\item \texttt{recvfrom($FD$, $Domain$) = $Message$}\index{\texttt{recvfrom/2}}: This is used for unreliable communication.  The \texttt{recvfrom}\index{\texttt{recvfrom/2}} function blocks until data is received, and returns the a map that contains the message that was received.
\item \texttt{recvfrom($FD$, $Flags$, $Domain$) = $Message$}\index{\texttt{recvfrom/3}}:  The $Flags$ parameter is a list of options.  The options can be \texttt{oob}, \texttt{peek}, \texttt{waitall(Length)}, and \texttt{dontwait}.
\item \texttt{close($FD$)}\index{\texttt{close/1}}
\end{itemize}

\section{\label{unix_domain}Communication on the Unix Domain}
The Unix domain\index{unix domain} is used for inter-process\index{process} communication on the Unix operating system.  The processes\index{process} are located on the same host.  The following is an example of communication in the Unix domain\index{unix domain}.

\subsection*{Example}
\begin{verbatim}
import socket.

server(Name) =>
     FD = socket(unix, stream),
     bind(FD, unix, Name),
     listen(FD), % wait for connection
     Client = accept(FD),
     hello_to_client(Client),
     close(FD).

client(Name) =>
     FD = socket(unix, stream),
     connect(FD, unix, Name),
     hello_to_server(FD),
     close(FD). 

hello_to_client(Client) => % Server code to communicate with client
     % Begin communication
     send(Client, "Enter Input:"),
     Str = recv(Client), % wait for client input
     send(Client, Str),
     % End communication
     close(Client).

hello_to_server(FD) => % Client code to communicate with server
     % Begin communication
     Str = recv(FD), % wait for server's message
     println(Str)
     send(FD, "Hello"),
     Str := recv(FD),
     println(Str).
     % End communication
\end{verbatim}

The server and the client perform the same steps as they do in the TCP\index{tcp} example.  They can also communicate over UDP\index{udp}.   In this example, the server only listens for one request, so it does not fork a child process\index{child process}, and when the server completes, it closes its file descriptor.  This example is similar to the first example of the chapter.  

Communication between a server and a client in the Unix domain\index{unix domain} uses the following functions and predicates: 

\begin{itemize}
\item \texttt{socket($Domain$, $Type$) = $FD$}\index{\texttt{socket/2}}: The file descriptor $FD$ represents a communication endpoint, which is a process\index{process} in the Unix domain\index{unix domain}.
\item \texttt{unix\_socket() = $FD$}\index{\texttt{unix\_socket/0}}: This performs the same operation as \texttt{socket(unix, stream)}\index{\texttt{socket/2}}.
\item \texttt{bind($FD$, $Unix$, $Name$)}\index{\texttt{bind/3}}: Since the processes\index{process} are located on the same host, the sockets\index{socket} do not need to be identified by addresses and ports in the \texttt{bind}\index{\texttt{bind/3}} predicate.  Instead, Unix domain\index{unix domain} sockets\index{socket} are identified by a file name in the file system.  Therefore, the process\index{process} is bound to a file name using \texttt{bind($FD$, unix, $Name$)}\index{\texttt{bind/3}}.  The \texttt{unix} atom is the only value that is allowed for the $Unix$ parameter.
\item \texttt{unix\_bind($FD$, $Name$)}\index{\texttt{unix\_bind/2}}: This performs the same operation as \texttt{bind($FD$, unix, $Name$)}\index{\texttt{bind/3}}.
\item \texttt{listen($FD$, $Backlog$)}\index{\texttt{listen/2}}
\item \texttt{listen($FD$)}\index{\texttt{listen/1}}
\item \texttt{accept($FD$) = $Client$}\index{\texttt{accept/1}}
\item \texttt{connect($FD$, $Unix$, $Name$)}\index{\texttt{connect/3}}: Since the processes\index{process} are located on the same host, the sockets\index{socket} do not need to be identified by addresses and ports in the \texttt{connect}\index{\texttt{connect/3}}  predicate.  Instead, Unix domain\index{unix domain} sockets\index{socket} are identified by a file name in the file system.  Therefore, the client process\index{process} connects to the server process\index{process} by using the server process's\index{process} file name in the \texttt{connect($FD$, $Unix$, $Name$)}\index{\texttt{connect/3}} predicate. The \texttt{unix} atom is the only value that is allowed for the $Unix$ parameter.
\item \texttt{unix\_connect($FD$, $Name$)}\index{\texttt{unix\_connect/2}}: This performs the same operation as \texttt{connect(FD, unix, Name)}\index{\texttt{connect/3}}.
\item \texttt{send($FD$, $Message$) = $NBytes$}\index{\texttt{send/2}}: This is used for reliable communication.  This function blocks until it is able to send the message.  The \texttt{send}\index{\texttt{send/2}} function returns the number of bytes that were sent.
\item \texttt{send($FD$, $Message$, $Flags$) = $NBytes$}\index{\texttt{send/3}}: The $Flags$ parameter is a list of options. The options can be \texttt{oob}, \texttt{dontroute}, \texttt{dontwait}, and \texttt{nosignal}.
\item \texttt{recv($FD$) = $Message$}\index{\texttt{recv/1}}: This is used for reliable communication.  It blocks until data arrives.  The \texttt{recv}\index{\texttt{recv/1}} function returns a message string.  
\item \texttt{recv($FD$, $Flags$) = $Message$}\index{\texttt{recv/2}}:  The $Flags$ parameter is a list of options.   The options can be \texttt{oob}, \texttt{peek}, \texttt{waitall(Length)}, and \texttt{dontwait}.
\item \texttt{sendto($FD$, $Message$, $Name$) = $NBytes$}\index{\texttt{sendto/3}}: This is used for unreliable communication.  This function blocks until it is able to send the message.  Note that \texttt{sendto}\index{\texttt{sendto/3}} in the Unix domain\index{unix domain} uses the name of a file.  The \texttt{sendto}\index{\texttt{sendto/3}} function returns the number of bytes that were sent.  
\item \texttt{sendto($FD$, $Message$, $Flags$, $Name$) = $NBytes$}\index{\texttt{sendto/4}}: The $Flags$ parameter is a list of options. The options can be \texttt{oob}, \texttt{dontroute}, \texttt{dontwait}, and \texttt{nosignal}.
\item \texttt{recvfrom($FD$, $Domain$) = $Message$}\index{\texttt{recvfrom/2}}: This is used for unreliable communication.  In the Unix domain\index{unix domain}, the \texttt{recvfrom}\index{\texttt{recvfrom/2}} function returns a map with the keys \texttt{name}, and \texttt{message}.  The key \texttt{name} is a string that stores the file name to which the process\index{process} is bound.  The key \texttt{message} contains the actual message string that was received.  
\item \texttt{recvfrom($FD$, $Flags$, $Domain$) = $Message$}\index{\texttt{recvfrom/3}}:  The $Flags$ parameter is a list of options.  The options can be \texttt{oob}, \texttt{peek}, \texttt{waitall(Length)}, and \texttt{dontwait}.
\item \texttt{close($FD$)}\index{\texttt{close/1}}
\end{itemize}

\section{Other Socket Functions and Predicates}
\subsection{\label{socket_options}Socket Options}
Picat allows users to access and to modify socket\index{socket} options.  For example, the Multicasting\index{multicast} section showed how to modify a socket's\index{socket} options to allow it to join or to leave a group.  The following example shows how to read a few of a socket's\index{socket} default attributes.

\subsection*{Example}
\begin{verbatim}
import socket.

show_attributes =>
     println("TCP/IP socket"),
     attributes(inet, stream),
     println("UDP/IP socket"),
     attributes(inet, dgram),
     println("TCP/IPv6 socket"),
     attributes(inet6, stream).

attributes(Domain, Type) =>
     FD = socket(Domain, Type),
     Live = getsockopt(FD, socket, keepalive),
     println(Val),
     OOB = getsockopt(FD, socket, oobinline),
     println(OOB),
     Delay = getsockopt(tcp, nodelay),
     println(Delay),
     Multicast = getsockopt(ipv6, multicasthops),
     println(Multicast),
     close(FD). 
\end{verbatim}

The following functions are used to access and to modify the socket\index{socket} options:
\begin{itemize}
\item \texttt{setsockopt($FD$, $Level$, $Option$, $Value$)}\index{\texttt{setsockopt/4}} 
\item \texttt{getsockopt($FD$, $Level$, $Option$) = $Value$}\index{\texttt{getsockopt/3}}
\end{itemize}
The $Level$ parameter is an atom indicating the protocol level at which the $Option$ parameter is defined.  The $Level$ parameter can be one of the following atoms: \texttt{socket}, \texttt{tcp}, \texttt{ipx}, \texttt{ip}, or \texttt{ipv6}.  The $Option$ parameter is also an atom.  For a list of available options for each level, see Appendix \ref{chapter:sockopt}.  Note that some options can only be used in \texttt{getsockopt}\index{\texttt{getsockopt/3}}, and that the \texttt{getsockopt}\index{\texttt{getsockopt/3}} function returns the string ``Non-existent" if it does not recognize the $Option$ parameter.

\subsection{Host Information}
Picat provides other socket\index{socket} functions that allow users to extract information about Internet hosts.  The following example shows how to connect to \texttt{www.probp.com} by using \texttt{gethostbyname}\index{\texttt{gethostbyname/1}}.

\subsection*{Example}
\begin{verbatim}
import socket.

probp(Port) =>
     FD = socket(inet, stream),
     Server = gethostbyname("http://www.probp.com")
     if length(Server.addresses) > 0 then 
        Address = Server.addresses[1],
        connect(FD, inet, Address, Port),
        hello_to_server(FD) % Communicate with the server
     end,
     close(FD). 
\end{verbatim}

The following functions are used to extract Internet host information:
\begin{itemize}
\item \texttt{gethostbyname($Name$) = $Host$}\index{\texttt{gethostbyname/1}}
\item \texttt{gethostbyaddr($Addr$) = $Host$}\index{\texttt{gethostbyaddr/1}}
\end{itemize}
These functions perform a DNS query.  They return a map with the keys \texttt{names} and \texttt{hosts}.  The key \texttt{names} is a list of names for the host, and the key \texttt{addresses} is a list of IP\index{ip} addresses by which the host is accessible.  If the query does not have any results, then the lists will be empty.

In the \texttt{gethostbyname}\index{\texttt{gethostbyname/1}} and \texttt{gethostbyaddr}\index{\texttt{gethostbyaddr/1}} functions, the parameter can be replaced by the atom \texttt{localhost}, in order to extract information about the current machine.

Picat provides two other functions for querying the canonical name and a single address for a host.  The following example shows how to connect to \texttt{www.probp.com} by using \texttt{getaddr}\index{\texttt{getaddr/1}}.

\subsection*{Example}
\begin{verbatim}
import socket.

probp(Port) =>
     FD = socket(inet, stream),
     Address = getaddr("http://www.probp.com"),
     connect(FD, inet, Address, Port),
     hello_to_server(FD), % Communicate with the server
     close(FD). 
\end{verbatim}

The following functions are used to extract a single name or address for a host:
\begin{itemize}
\item \texttt{getaddr($Name$) = $Addr$}\index{\texttt{getaddr/1}}
\item \texttt{getcanonicalname($Addr$) = $Name$}\index{\texttt{getcanonicalname/1}}
\end{itemize}
In the \texttt{getaddr}\index{\texttt{getaddr/1}} and \texttt{getcanonicalname}\index{\texttt{getcanonicalname/1}} functions, the parameter can be replaced by the atom \texttt{localhost}, in order to extract information about the current machine.  Unlike the \texttt{gethostbyname}\index{\texttt{gethostbyname/1}} and \texttt{gethostbyaddr}\index{\texttt{gethostbyaddr/1}} functions, if the query does not return any results, then the \texttt{getaddr}\index{\texttt{getaddr/1}} and \texttt{getcanonicalname}\index{\texttt{getcanonicalname/1}} functions will throw an error.  

\subsection{Services}
Some services, like Telnet and FTP, have pre-defined port numbers.  The following functions allow users to look up information about a service, such as the port number that it uses:
\begin{itemize}
\item \texttt{getservbyname($Name$) = $Service$}\index{\texttt{getservbyname/1}}: This function matches a service for any protocol.  It returns a map, with the keys \texttt{name}, \texttt{aliases}, \texttt{port}, and \texttt{protocol}.  The key \texttt{name} is the name of the service.  The key \texttt{aliases} is a list of alternative names for the service.  The key \texttt{port} is the port number to use, in order to access the service.  The key \texttt{protocol} is a string, indicating the protocol to use in order to access the service.
\item \texttt{getservbyname($Name$, $Type$) = $Service$}\index{\texttt{getservbyname/2}}: This function matches a service for a specific protocol.  The $Type$ parameter is either the atom \texttt{tcp} or the atom \texttt{udp}.
\item \texttt{getservport($Name$) = $Port$}\index{\texttt{getservport/1}}: This function returns the port number to use in order to access the service. 
\end{itemize}
These functions throw an error if the service cannot be found.

\ignore{
\end{document}
}
