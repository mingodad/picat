\documentclass[11pt]{report}
\textwidth 13.7cm
\textheight 21.5cm

\RequirePackage[T1]{fontenc}
\usepackage{chngpage}
\usepackage{makeidx}
\usepackage{multicol}
\usepackage{sectsty}
\usepackage{setspace}
\usepackage{times}
\usepackage{url}
\usepackage{vmargin}

\newcommand{\ignore}[1]{}

\sectionfont{\normalsize\bf}
\subsectionfont{\normalsize\bf}

\newcommand{\myimp}{\verb+ :- +}
\def\definitionname{Definition}

\makeindex

\usepackage[bookmarks=true,hidelinks]{hyperref}
\hypersetup{pdfstartview={XYZ null null 1.00}}

\begin{document}
\vspace*{4cm}
\begin{center}
{\Huge\bf A User's Guide to Picat} \\
{\large\bf Version 3.6} \\


\vspace*{1cm}
%\vspace*{8cm}

{\large\bf Neng-Fa Zhou and Jonathan Fruhman} \\
\vspace*{1cm}
{\bf Copyright \copyright \url{picat-lang.org}, 2013-2023.} \\
%{\bf Under construction}
{\bf Last updated January 9, 2024} \\
\end{center}
\thispagestyle{empty}
\clearpage

\pagestyle{plain}
\pagenumbering{roman}

\section*{Preface}
Picat is a general-purpose language that incorporates features from logic programming, functional programming, constraint programming,  and scripting languages. The letters in the name summarize Picat's features:

\begin{itemize}
\item \textbf{P}attern-matching: A \emph{predicate} defines a relation, and can have zero, one, or multiple answers. A \emph{function} is a special kind of a predicate that always succeeds with \emph{one} answer. Picat is a rule-based language. Predicates and functions are defined with pattern-matching rules. Since version 3.0, Picat also supports Prolog-style Horn clauses and Definite Clause Grammar (DCG) rules.

\item \textbf{I}ntuitive: Picat provides assignment and loop statements for programming everyday things. An assignable variable mimics multiple logic variables, each of which holds a value at a different stage of computation. Assignments are useful for computing aggregates and are used with the {\tt foreach} loop for implementing list and array comprehensions.

\item \textbf{C}onstraints: Picat supports constraint programming.  Given a set of variables, each of which has a domain of possible values, and a set of constraints that limit the acceptable set of assignments of values to variables, the goal is to find an assignment of values to the variables that satisfies all of the constraints. Picat provides four solver modules: {\tt cp}, {\tt sat}, {\tt smt}, and {\tt mip}. These four modules follow the same interface, which allows for seamless switching from one solver to another.

\item \textbf{A}ctors: Actors are event-driven calls.  Picat provides \emph{action rules} for describing event-driven behaviors of actors. Events are posted through channels. An actor can be attached to a channel in order to watch and to process its events. All the propagators used in {\tt cp} are implemented as actors.
% Picat treats threads as channels, and allows the use of action rules\index{action rule} to program concurrent threads\index{thread}.

\item \textbf{T}abling: Tabling can be used to store the results of certain calculations in memory, allowing the program to do a quick table lookup instead of repeatedly calculating a value. As computer memory grows, tabling is becoming increasingly important for offering dynamic programming solutions for many problems. The \texttt{planner} module, which is implemented by the use of tabling, has been shown to be an efficient tool for solving planning problems.

\end{itemize}

The support of unification, non-determinism, tabling, and constraints makes Picat more suitable than functional and scripting languages for symbolic computations. Picat is more convenient than Prolog for scripting and modeling. With arrays, loops, and comprehensions, it is not rare to find problems for which Picat requires an order of magnitude fewer lines of code to describe than Prolog. Picat is more scalable than Prolog. The use of pattern-matching rather than unification facilitates indexing of rules. Picat is not as powerful as Prolog for metaprogramming and it's impossible to write a meta-interpreter for Picat in Picat itself. Nevertheless, this weakness can be remedied with library modules for implementing domain-specific languages.

The Picat implementation is based on the B-Prolog engine. The current implementation is ready for many kinds of applications. It also serves as a foundation for new additions. The project is open, and you are welcome to join as a developer, a sponsor, a user, or a reviewer. Please contact \url{picat@picat-lang.org} and join the news group \url{https://groups.google.com/g/picat-lang}.




\clearpage
\section*{License}
The copyright of Picat is owned by \url{picat-lang.org}. Picat is provided, free of charge, for any purposes, including commercial ones. The C source files of Picat are covered by the Mozilla Public License, v. 2.0 (\url{http://mozilla.org/MPL/2.0/}). In essence, anyone is allowed to build works, including proprietary ones, based on Picat, as long as the Source Code Form is retained. The copyright holders, developers, and distributors will not be held liable for any direct or indirect damages.

 
\section*{Acknowledgements}
The initial design of Picat was published in December 2012, and the first alpha version was released in May 2013.  Many people have contributed to the project by reviewing the ideas, the design, the implementation, and/or the documentation, including Roman Bart\'{a}k, Nikhil Barthwal, Mike Bionchik, Lei Chen, Veronica Dahl, Claudio Cesar de S\'{a}, Agostino Dovier, Sergii Dymchenko, Julio Di Egidio, Christian Theil Have, H{\aa}kan Kjellerstrand,  Annie Liu, Nuno Lopes, Marcio Minicz, Richard O'Keefe, Lorenz Schiffmann, Paul Tarau, and Jan Wielemaker.  Special thanks to H{\aa}kan Kjellerstrand, who has been programming in Picat and blogging about Picat since May 2013. The system wouldn't have matured so quickly without H{\aa}kan's hundreds of programs (\url{http://hakank.org/}). Thanks also go to Bo Yuan (Bobby) Zhou, who designed the \url{picat-lang.org} web page, Sanders Hernandez, who implemented the interface to the FANN neural network library, and Domingo Alvarez Duarte, who ported Picat to MinGW (\url{https://github.com/mingodad/picat}). The Picat project was supported in part by the NSF under grant numbers CCF1018006 and CCF1618046.

The Picat implementation is based on the B-Prolog engine. It uses the following public domain modules: \ignore{\texttt{prism} by Taisuke Sato and Yoshitaka Kameya; }\texttt{token.c} by Richard O'Keefe; \texttt{getline.c} by Chris Thewalt; \texttt{bigint.c} by Matt McCutchen; \texttt{Espresso} (by Berkeley); \texttt{FANN} by Steffen Nissen. In addition, Picat also provides interfaces to the SAT solver Kissat (\url{https://github.com/arminbiere/kissat}), \texttt{Gurobi} by Gurobi Optimization, Inc, \texttt{CBC} by John Forrest, \texttt{GLPK} by Andrew Makhorin, \texttt{SCIP} by the Zuse Institute, and \texttt{Z3} by Microsoft.

\tableofcontents

\cleardoublepage
\pagestyle{plain}
\setcounter{page}{1}
\pagenumbering{arabic}

\input{overview.tex}
\input{intro_sys.tex}
\input{data_types.tex}
\input{predfunc.tex}
\input{loops.tex}
\input{exception.tex}
\input{tabling.tex}
\ignore{
\documentstyle[11pt]{report}
\textwidth 13.7cm
\textheight 21.5cm
\newcommand{\myimp}{\verb+ :- +}
\newcommand{\ignore}[1]{}
\def\definitionname{Definition}

\makeindex
\begin{document}
}
\chapter{\label{chapter:planner}The \texttt{planner} Module}
The \texttt{planner} module provides several predicates for solving planning problems. Given an initial state, a final state, and a set of possible actions, a planning problem is to find a plan that transforms the initial state to the final state. In order to use the \texttt{planner} module to solve a planning problem, users have to provide the condition for the final states and the state transition diagram through the following global predicates:\index{planning}\index{planner}
\begin{itemize}
\item \texttt{final($S$)}\index{\texttt{final/1}}: This predicate succeeds if $S$ is a final state.
\item \texttt{final($S$,$Plan$,$Cost$)}\index{\texttt{final/3}}: A final state can be reached from $S$ by the action sequence in $Plan$ with $Cost$. If this predicate is not given, then the system assumes the following definition:
\begin{verbatim}
    final(S,Plan,Cost) => Plan=[], Cost=0, final(S).
\end{verbatim}

\item \texttt{action($S$,$NextS$,$Action$,$ActionCost$)}\index{\texttt{action/4}}: This predicate encodes the state transition diagram of the planning problem. The state $S$ can be transformed into $NextS$ by performing $Action$. The cost of $Action$ is $ActionCost$. If the plan's length is the only interest, then $ActionCost$ should be 1.
\end{itemize}

A state is normally a ground term. As all states are tabled during search, it is of paramount importance to find a good representation for states such that terms among states can be shared as much as possible.

In addition to the two required predicates given above, users can optionally provide the following global procedures to assist Picat in searching for plans:
\begin{itemize}
\item \texttt{heuristic($S$)}\index{\texttt{heuristic/1}}: This function returns an estimation of the resource amount needed to transform $S$ into a goal state. This estimation is said to be \textit{admissible} if it never exceeds the real cost. All heuristic estimations must be admissible in order for the planner to find optimal plans. This function is used by the planner to check the heuristic estimation before each state expansion.

\item \texttt{sequence($P$,$Action$)}\index{\texttt{sequence/2}}: This predicate binds $Action$ to a viable action form based on the current partial plan $P$. Note that the actions in list $P$ are in reversed order, with the most recent action occurring first in the list, and the first action occurring last in the list. The planner calls \texttt{sequence/2} to find an action for expanding the current state before calling \texttt{action/4}. For example,
\begin{verbatim}
sequence([move(R,X,Y)|_], Action) ?=> Action = $move(R,Y,_).
sequence([move(R,_,_)|_], Action) ?=> Action = $jump(R).
sequence([move(R,_,_)|_], Action) => Action = $wait(R).
sequence(_, _) => true.
\end{verbatim}
These sequence rules ban robots from moving in an interleaving fashion; a robot must continue to move until it takes the action \texttt{jump} or \texttt{wait} before another robot can start moving. The last rule \texttt{sequence(\_, \_) => true} is necessary; it permits any action to be taken if the partial plan is empty, or if the most recent action in the partial plan is not \texttt{move}.
\end{itemize}

\section{Depth-Bounded Search}
Depth-bounded search amounts to exploring the search space, taking into account the current available resource amount. A new state is only explored if the available resource amount is non-negative. When depth-bounded search is used, the function {\tt current\_resource()} can be used to retrieve the current resource amount. If the heuristic estimate of the cost to travel from the current state to the final state is greater than the available resource amount, then the current state fails.

\begin{itemize}
\item \texttt{plan($S$,$Limit$,$Plan$,$Cost$)}\index{\texttt{plan/4}}: This predicate, if it succeeds, binds $Plan$ to a plan that can transform state $S$ to a final state that satisfies the condition given by {\tt final/1} or {\tt final/3}. $Cost$ is the cost of $Plan$, which cannot exceed $Limit$, which is a given non-negative integer. 

\item \texttt{plan($S$,$Limit$,$Plan$)}: If the second argument is an integer, then this predicate is the same as the \texttt{plan/4} predicate, except that the plan's cost is not returned. 

\item \texttt{plan($S$,$Plan$,$Cost$)}:\index{\texttt{plan/3}} If the second argument is a variable, then this predicate is the same as the \texttt{plan/4} predicate, except that the limit is assumed to be 268435455.

\item \texttt{plan($S$,$Plan$)}\index{\texttt{plan/2}}: This predicate is the same as the \texttt{plan/4} predicate, except that the limit is assumed to be 268435455, and that the plan's cost is not returned.

\item \texttt{best\_plan($S$,$Limit$,$Plan$,$Cost$)}\index{\texttt{best\_plan/4}}: This predicate finds an optimal plan by using the following algorithm:  It first calls {\tt plan/4} to find a plan of 0 cost. If no plan is found, then it increases the cost limit to 1 or double the cost limit. Once a plan is found, the algorithm uses binary search to find an optimal plan.

\item \texttt{best\_plan($S$,$Limit$,$Plan$)}: If the second argument is an integer, then this predicate is the same as the \texttt{best\_plan/4} predicate, except that the plan's cost is not returned. 

\item \texttt{best\_plan($S$,$Plan$,$Cost$)}\index{\texttt{best\_plan/3}}: If the second argument is a variable, then this predicate is the same as the \texttt{best\_plan/4} predicate, except that the limit is assumed to be 268435455.

\item \texttt{best\_plan($S$,$Plan$)}\index{\texttt{best\_plan/2}}: This predicate is the same as the \texttt{best\_plan/4} predicate, except that the limit is assumed to be 268435455, and that the plan's cost is not returned.

\item \texttt{best\_plan\_nondet($S$,$Limit$,$Plan$,$Cost$)}\index{\texttt{best\_plan\_nondet/4}}: This predicate is the same as \\\texttt{best\_plan($S$,$Limit$,$Plan$,$Cost$)}, except that it allows multiple best plans to be returned through backtracking.

\item \texttt{best\_plan\_nondet($S$,$Limit$,$Plan$)}: If the second argument is an integer, then this predicate is the same as the \texttt{best\_plan\_nondet/4} predicate, except that the plan's cost is not returned. 

\item \texttt{best\_plan\_nondet($S$,$Plan$,$Cost$)}\index{\texttt{best\_plan\_nondet/3}}: If the second argument is a variable, then this predicate is the same as the \texttt{best\_plan\_nondet/4} predicate, except that the limit is assumed to be 268435455.

\item \texttt{best\_plan\_nondet($S$,$Plan$)}\index{\texttt{best\_plan\_nondet/2}}: This predicate is the same as the \texttt{best\_plan\_nondet/4} predicate, except that the limit is assumed to be 268435455, and that the plan's cost is not returned.

\item \texttt{best\_plan\_bb($S$,$Limit$,$Plan$,$Cost$)}\index{\texttt{best\_plan\_bb/4}}: 
This predicate, if it succeeds, binds $Plan$ to an optimal plan that can transform state $S$ to a final state. $Cost$ is the cost of $Plan$, which cannot exceed $Limit$, which is a given non-negative integer. The branch-and-bound algorithm is used to find an optimal plan.

\item \texttt{best\_plan\_bb($S$,$Limit$,$Plan$)}: If the second argument is an integer, then this predicate is the same as the \texttt{best\_plan\_bb/4} predicate, except that the plan's cost is not returned. 

\item \texttt{best\_plan\_bb($S$,$Plan$,$Cost$)}\index{\texttt{best\_plan\_bb/3}}: If the second argument is a variable, then this predicate is the same as the \texttt{best\_plan\_bb/4} predicate, except that the limit is assumed to be 268435455.

\item \texttt{best\_plan\_bb($S$,$Plan$)}\index{\texttt{best\_plan\_bb/2}}: 
This predicate is the same as the \texttt{best\_plan\_bb/4} predicate, except that the limit is assumed to be 268435455, and that the plan's cost is not returned.

\item \texttt{best\_plan\_bin($S$,$Limit$,$Plan$,$Cost$)}\index{\texttt{best\_plan\_bin/4}}: 
This predicate, if it succeeds, binds $Plan$ to an optimal plan that can transform state $S$ to a final state. $Cost$ is the cost of $Plan$, which cannot exceed $Limit$, which is a given non-negative integer. The branch-and-bound algorithm is used with binary search to find an optimal plan.

\item \texttt{best\_plan\_bin($S$,$Limit$,$Plan$)}: If the second argument is an integer, then this predicate is the same as the \texttt{best\_plan\_bin/4} predicate, except that the plan's cost is not returned. 

\item \texttt{best\_plan\_bin($S$,$Plan$,$Cost$)}\index{\texttt{best\_plan\_bin/3}}: If the second argument is a variable, then this predicate is the same as the \texttt{best\_plan\_bin/4} predicate, except that the limit is assumed to be 268435455.

\item \texttt{best\_plan\_bin($S$,$Plan$)}\index{\texttt{best\_plan\_bin/2}}: 
This predicate is the same as the \texttt{best\_plan\_bin/4} predicate, except that the limit is assumed to be 268435455, and that the plan's cost is not returned.

\item \texttt{current\_resource()=$Amount$}\index{\texttt{current\_resource/0}}: This function returns the current available resource amount of the current node. If the current execution path was not initiated by one of the calls that performs resource-bounded search, then 268435455 is returned. This function can be used to check the heuristics. If the heuristic estimate of the cost to travel from the current state to a final state is greater than the available resource amount, then the current state can be failed.

\item \texttt{current\_plan()=$Plan$}\index{\texttt{current\_plan/0}}: This function returns the current plan that has transformed the initial state to the current state. If the current execution path was not initiated by one of the calls that performs resource-bounded search, then [] is returned.

\item \texttt{current\_resource\_plan\_cost($Amount$,$Plan$,$Cost$)}\index{\texttt{current\_resource\_plan\_cost/3}}: This predicate retrieves the attributes of the current node in the search tree, including the resource amount, the path to the node, and its cost.

\item \texttt{is\_tabled\_state($S$)}\index{\texttt{is\_tabled\_state/1}}: This predicate succeeds if the state $S$ has been explored before and has been tabled.
\end{itemize}

\section{Depth-Unbounded Search}
In contrast to depth-bounded search, depth-unbounded search does not take into account the available resource amount. A new state can be explored even if no resource is available for the exploration. The advantage of depth-unbounded search is that failed states are never re-explored.

\begin{itemize}
\item \texttt{plan\_unbounded($S$,$Limit$,$Plan$,$Cost$)}\index{\texttt{plan\_unbounded/4}}: This predicate, if it succeeds, binds $Plan$ to a plan that can transform state $S$ to a final state. $Cost$ is the cost of $Plan$, which cannot exceed $Limit$, which is a given non-negative integer.

\item \texttt{plan\_unbounded($S$,$Limit$,$Plan$)} If the second argument is an integer, then this predicate is the same as the \texttt{plan\_unbounded/4} predicate, except that the plan's cost is not returned. 

\item \texttt{plan\_unbounded($S$,$Plan$,$Cost$)}\index{\texttt{plan\_unbounded/3}}: If the second argument is a variable, then this predicate is the same as the \texttt{plan\_unbounded/4} predicate, except that the limit is assumed to be 268435455.

\item \texttt{plan\_unbounded($S$,$Plan$)}\index{\texttt{plan\_unbounded/2}}: This predicate is the same as the above predicate, except that the limit is assumed to be 268435455.

\item \texttt{best\_plan\_unbounded($S$,$Limit$,$Plan$,$Cost$)}\index{\texttt{best\_plan\_unbounded/4}}: This predicate, if it succeeds, binds $Plan$ to an optimal plan that can transform state $S$ to a final state. $Cost$ is the cost of $Plan$, which cannot exceed $Limit$, which is a given non-negative integer.

\item \texttt{best\_plan\_unbounded($S$,$Limit$,$Plan$)} If the second argument is an integer, then this predicate is the same as the \texttt{best\_plan\_unbounded/4} predicate, except that the plan's cost is not returned.

\item \texttt{best\_plan\_unbounded($S$,$Plan$,$Cost$)}\index{\texttt{best\_plan\_unbounded/3}}: If the second argument is a variable, then this predicate is the same as the \texttt{best\_plan\_unbounded/4} predicate, except that the limit is assumed to be 268435455.

\item \texttt{best\_plan\_unbounded($S$,$Plan$)}\index{\texttt{best\_plan\_unbounded/2}}: This predicate is the same as the above predicate, except that the limit is assumed to be 268435455.
\end{itemize}

\section{Examples}
The program shown in Figure \ref{fig:farmer2} solves the Farmer's problem by using the \texttt{planner} module. 

\begin{figure}
\begin{center}
\begin{verbatim}
    import planner.

    go =>
        S0=[s,s,s,s],
        best_plan(S0,Plan),
        writeln(Plan).

    final([n,n,n,n]) => true.

    action([F,F,G,C],S1,Action,ActionCost) ?=>
        Action = farmer_wolf,
        ActionCost = 1,        
        opposite(F,F1),
        S1 = [F1,F1,G,C],
        not unsafe(S1).
    action([F,W,F,C],S1,Action,ActionCost) ?=>
        Action = farmer_goat,
        ActionCost = 1,        
        opposite(F,F1),
        S1 = [F1,W,F1,C],
        not unsafe(S1).
    action([F,W,G,F],S1,Action,ActionCost) ?=>
        Action = farmer_cabbage,
        ActionCost = 1,        
        opposite(F,F1),
        S1 = [F1,W,G,F1],
        not unsafe(S1).
    action([F,W,G,C],S1,Action,ActionCost) =>
        Action = farmer_alone,
        ActionCost = 1,        
        opposite(F,F1),
        S1 = [F1,W,G,C],
        not unsafe(S1).

    index (+,-) (-,+)
    opposite(n,s).
    opposite(s,n).

    unsafe([F,W,G,_C]), W == G, F !== W => true.
    unsafe([F,_W,G,C]), G == C, F !== G => true.
\end{verbatim}
\end{center}
\caption{\label{fig:farmer2}A program for the Farmer's problem using \texttt{planner}.}
\end{figure}

Figure \ref{fig:15puzzlesol} gives a program for the 15-puzzle problem. A state is represented as a list of sixteen locations, each of which takes the form \texttt{($R_i$,$C_i$)}, where $R_i$ is a row number and $C_i$ is a column number. The first element in the list gives the position of the empty square, and the remaining elements in the list give the positions of the numbered tiles from 1 to 15. The function \texttt{heuristic(Tiles)} returns the Manhattan distance between the current state and the final state.

\begin{figure}[t]
\begin{center}
\begin{verbatim}
import planner.

main =>
    InitS = [(1,2),(2,2),(4,4),(1,3),
             (1,1),(3,2),(1,4),(2,4),
             (4,2),(3,1),(3,3),(2,3),
             (2,1),(4,1),(4,3),(3,4)],
    best_plan(InitS,Plan),
    foreach (Action in Plan)
       println(Action)
    end.

final(State) => State = [(1,1),(1,2),(1,3),(1,4),
                         (2,1),(2,2),(2,3),(2,4),
                         (3,1),(3,2),(3,3),(3,4),
                         (4,1),(4,2),(4,3),(4,4)].

action([P0@(R0,C0)|Tiles],NextS,Action,Cost) =>
    Cost = 1,
    (R1 = R0-1, R1 >= 1, C1 = C0, Action = up;
     R1 = R0+1, R1 =< 4, C1 = C0, Action = down;
     R1 = R0, C1 = C0-1, C1 >= 1, Action = left;
     R1 = R0, C1 = C0+1, C1 =< 4, Action = right),
    P1 = (R1,C1),
    slide(P0,P1,Tiles,NTiles),
    NextS = [P1|NTiles].

% slide the tile at P1 to the empty square at P0
slide(P0,P1,[P1|Tiles],NTiles) =>
    NTiles = [P0|Tiles].
slide(P0,P1,[Tile|Tiles],NTiles) =>
    NTiles = [Tile|NTilesR],
    slide(P0,P1,Tiles,NTilesR).

heuristic([_|Tiles]) = Dist =>
    final([_|FTiles]),
    Dist = sum([abs(R-FR)+abs(C-FC) : 
                {(R,C),(FR,FC)} in zip(Tiles,FTiles)]).
\end{verbatim}
\end{center}
\caption{\label{fig:15puzzlesol}A program for the 15-puzzle}
\end{figure}




\input{module.tex}
\input{io.tex}
\input{ar.tex}
%\input{thread.tex}
%\input{process.tex}
\input{constraints.tex}
\ignore{
\documentstyle[11pt]{report}
\textwidth 13.7cm
\textheight 21.5cm
\newcommand{\myimp}{\verb+ :- +}
\newcommand{\ignore}[1]{}
\def\definitionname{Definition}

\makeindex
\begin{document}
}

\chapter{\label{chapter:nn}The \texttt{nn} (Neural Networks) Module}

The \texttt{nn} module provides a high-level interface between Picat and the FANN\footnote{\url{http://leenissen.dk/fann/wp/}} neural networks library, which implements feedforward neural networks.\footnote{The Picat-FANN interface was implemented by Sanders Hernandez.}\index{neural network}\index{feedforward neural network}  A \emph{feedforward neural network} consists of \emph{neurons} organized in layers from an input layer to an output layer, possibly with a number of hidden layers. A feedforward network represents a function from input to output. Neurons in a layer (except for the input layer) are connected to neurons in the previous layer. The connections have weights.  The neurons in the input layer receive the input. The information is propagated forward through the layers until it reaches the output layer, where the output is returned. The information that a neuron receives is determined by the connected predecessor neurons, the weights of the connections, and an \emph{activation function}.\index{activation function} The connection weights of a neural network are normally adjusted through training on a given set of input-output pairs, called \emph{training data}. Once a neural network is trained, it can be used to predict the output for a given input.

The following gives an example program which creates a neural network for the \texttt{xor} function,  and trains it on a set of data stored in a file:
\begin{verbatim}
import nn.

main =>
    NN = new_nn({2,3,1}),
    nn_train(NN,"xor.data"),
    nn_save(NN,"xor.net"),
    nn_destroy_all.
\end{verbatim}
The function \texttt{new\_nn(\{2,3,1\})} returns a neural network with three layers, where the input layer has 2 neurons, the hidden layer has 3 neurons, and the output layer has 1 neuron. The program does not specify any activation functions used between layers, entailing that the default activation function, which is \texttt{sym\_sigmoid}, will be used. The predicate \texttt{nn\_train(NN,"xor.data")} trains the neural network with the training data stored in the file \texttt{"xor.data"}. The user is able to specify an algorithm to be used in the training and several parameters that affect the behavior of the algorithm, such as the maximum number of iterations (called \textit{epochs}), the learning rate, and the error function. This example does not specify a training algorithm or any of the training parameters, entailing that the default algorithm, which is \texttt{rprop}, is used with the default setting. The predicate \texttt{nn\_save} saves the trained neural network into a file named \texttt{"xor.net"}. The predicate \texttt{nn\_destroy\_all} clears the neural network and the internal data structures used during training.

The text file \texttt{"xor.data"} contains the following training data:
\begin{verbatim}
4 2 1
-1 -1
-1
-1 1
1
1 -1
1
1 1
-1
\end{verbatim}
The three integers in the first line state, respectively, that the number of input-output pairs is 4, the number of input values is 2, and the number of output values is 1. The remaining lines give the input-output pairs. 

The following program performs the same task as the above program, except that it trains the neural network with internal data:
\begin{verbatim}
import nn.

main =>
    NN = new_nn({2,3,1}),
    nn_train(NN,[({-1,-1}, -1), 
                 ({-1,1}, 1), 
                 ({1,-1}, 1), 
                 ({1,1}, -1)]),
    nn_save(NN,"xor.net"),
    nn_destroy_all.
\end{verbatim}
The predicate \texttt{nn\_train} is overloaded. When the second argument is a file name, Picat reads training data from the file. Otherwise, Picat expects a collection (a list or an array) of input-output pairs.

The following example program illustrates how to use a trained network:
\begin{verbatim}
import nn.

main =>
    NN = nn_load("xor.net"),
    printf("xor(-1,1) = %w\n",nn_run(NN,{-1,1})),
    nn_destroy_all.
\end{verbatim}
The function \texttt{nn\_load} loads a neural network. The function \texttt{nn\_run} uses the network to predict the output for an input.

\section{Create, Print, and Destroy Neural Networks}
\begin{itemize}
\item \texttt{new\_nn($Layers$) = $NN$}\index{\texttt{new\_nn/1}}: This function creates a fully-connected neural network of the structure $Layers$, which is a collection of positive integers indicating the number of neurons in each layer.

\item \texttt{new\_standard\_nn($Layers$) = $NN$}\index{\texttt{new\_standard\_nn/1}}: This function is the same as \texttt{new\_nn($Layers$)}.

\item \texttt{new\_sparse\_nn($Layers$,$Rate$) = $NN$}\index{\texttt{new\_sparse\_nn/2}}: This function creates a sparse neural network that has the structure $Layers$ and the connection rate $Rate$. The connection rate determines the sparseness of the network, with 1 indicating that the network is fully connected, and 0 indicating that the network is not connected at all.

\item \texttt{new\_sparse\_nn($Layers$) = $NN$}\index{\texttt{new\_sparse\_nn/1}}: This function is the same as \texttt{new\_sparse\_nn($Layers$, 0.5)}.

\item \texttt{nn\_print($NN$)}\index{\texttt{nn\_print/1}}: This predicate prints the attributes of the neural network $NN$, including the connections, the weights, the activation functions, and some other parameters.

\item \texttt{nn\_destroy($NN$)}\index{\texttt{nn\_destroy/1}}: This predicate destroys the neural network $NN$.

\item \texttt{nn\_destroy\_all}\index{\texttt{nn\_destroy\_all/0}}: This predicate destroys all the neural networks and the internal data structures.
\end{itemize}

\section{Activation Functions}
An activation function for a neuron determines how information is propagated to it from its predecessor neurons. When a new neural network is created, it uses the default activation function \texttt{sym\_sigmoid} for all of its non-input neurons. The following predicates can be utilized to set activation functions.

\begin{itemize}
\item \texttt{nn\_set\_activation\_function\_layer($NN$,$Func$,$Layer$)}\index{\texttt{nn\_set\_activation\_.../3}}: This predicate sets the activation function to $Func$ for all the neurons in layer $Layer$ in the neural network $NN$. Let the number of layers in $NN$ be $n$. Since the first layer is numbered 1, $Layer$ must satisfy $2\ \le\ Layer\ \le n$. The following activation functions are available:
\begin{itemize}
\item \texttt{linear}
\item \texttt{threshold}
\item \texttt{sym\_threshold}: symmetric threshold.
\item \texttt{sigmoid}
\item \texttt{step\_sigmoid}: stepped sigmoid
\item \texttt{sym\_sigmoid}: symmetric sigmoid
\item \texttt{elliot}: an alternative for sigmoid
\item \texttt{sym\_elliot}: symmetric elliot
\item \texttt{gaussian}
\item \texttt{sym\_gaussian}: symmetric Gaussian
\item \texttt{linear\_piece}
\item \texttt{sym\_linear\_piece}: symmetric linear piece
\item \texttt{sin}
\item \texttt{sym\_sin}: symmetric sin
\item \texttt{cos}
\item \texttt{sym\_cos}: symmetric cos
\end{itemize}
The detault activation function is \texttt{sym\_sigmoid}. Each of these functions has a corresponding name in FANN. Please refer to the FANN documentation for a more detailed description of these functions.

\item \texttt{nn\_set\_activation\_function\_hidden($NN$,$Func$)}\index{\texttt{nn\_set\_activation\_.../2}}: This predicate sets the activation function to $Func$ for all of the hidden layers in the neural network $NN$.

\item \texttt{nn\_set\_activation\_function\_output($NN$,$Func$)}\index{\texttt{nn\_set\_activation\_.../2}}: This predicate sets the activation function to $Func$ for the output layer in the neural network $NN$.

\item \texttt{nn\_set\_activation\_steepness\_layer($NN$,$Steepness$,$Layer$)}\index{\texttt{nn\_set\_activation\_.../3}}: This predicate sets the activation steepness for all of the neurons in $Layer$, where $-1\ \le\ Steepness\ \le 1$ and $2 \le Layer\ \le n$ ($n$ is the number of layers in $NN$).  A high steepness value means a more aggressive training. The default steepness is 0.5.

\item \texttt{nn\_set\_activation\_steepness\_hidden($NN$,$Steepness$)}\index{\texttt{nn\_set\_activation\_.../2}}: This predicate sets the activation steepness to $Steepness$ for all of the hidden layers in the neural network $NN$.

\item \texttt{nn\_set\_activation\_steepness\_output($NN$,$Steepness$)}\index{\texttt{nn\_set\_activation\_.../2}}: This predicate sets the activation steepness to $Steepness$ for the output layer in the neural network $NN$.
\end{itemize}

\section{Training Data}
A training dataset can be supplied to FANN either through a text file or a Picat collection. A training dataset file must have the following format:

\begin{verbatim}
num_train_data num_input num_output
input_data separated by space
output_data separated by space
.
.
.
input_data separated by space
output_data separated by space
\end{verbatim}
A training dataset stored in a Picat collection must be either a list or an array of input-output pairs. An input-output pair has the form \texttt{($Input$,$Output$)}, where $Input$ is an array of numbers or a single number, and so is $Output$.

\begin{itemize}
\item \texttt{nn\_train\_data\_size($Data$) = $Size$}\index{\texttt{nn\_train\_data\_size/1}}: This function returns the number of input-output pairs in the dataset $Data$, which is either a file name or a Picat collection.

\item \texttt{nn\_train\_data\_get($Data$,$I$) = $Pair$}\index{\texttt{nn\_train\_data\_get/2}}: This function returns the $I$th pair in the dataset $Data$. Notice that while this function takes linear time when $Data$ is a list, it takes constant time when $Data$ is a file name or an array.\footnote{The data is loaded into an internal array when the dataset is accessed the first time.}

\item \texttt{nn\_train\_data\_load($File$) = $Data$}\index{\texttt{nn\_train\_data\_load/1}}: This function loads a dataset stored in $File$ into a Picat array, and returns the array.

\item \texttt{nn\_train\_data\_save($Data$,$File$)}\index{\texttt{nn\_train\_data\_save/2}}: This predicate saves a dataset in a Picat collection $Data$ into a file named $File$.
\end{itemize}

\section{Train Neural Networks}
\begin{itemize}
\item \texttt{nn\_train($NN$,$Data$,$Opts$)}\index{\texttt{nn\_train/3}}: This predicate trains the neural network $NN$ using the dataset $Data$ under the control of training options $Opts$. The following training options are supported:
\begin{itemize}
\item \texttt{maxep($X$)}: $X$ is the maximum number of epochs for training.
\item \texttt{report($X$)}:	reporting every $X$ number of epochs. 
\item \texttt{derror($X$)}:	$X$ is the desired error in training.
\item \texttt{train\_func($X$)}: $X$ is the training algorithm to be used, which is one of the following: \texttt{batch}, \texttt{inc}, \texttt{qprop} (quick prop), \texttt{rprop}, and \texttt{sprop} (sarprop).
\item \texttt{lrate($X$)}: $X$ is the learning rate.
\item \texttt{momentum($X$)}: $X$ is the learning momentum.
\item \texttt{error\_func($X$)}: $X$ is the error function to be used, which is either \texttt{linear} or \texttt{tanh}.
\item \texttt{stop\_func($X$)}:	$X$ is the stop function to be used, which is either \texttt{bit} or \texttt{mse}.
\item \texttt{bfl($X$)}:  $X$ is the bit fail limit.
\item \texttt{qp\_decay($X$)}:	 $X$ is the \texttt{qprop} decay.
\item \texttt{qp\_mu($X$)} : $X$ is \texttt{qprop}  mu factor.
\item \texttt{rp\_increase($X$)}:	$X$ is the \texttt{rprop} increase factor.
\item \texttt{rp\_decrease($X$)}: $X$ is the \texttt{rprop} decrease factor.
\item \texttt{rp\_deltamin($X$)}: $X$ is the \texttt{rprop} delta min.
\item \texttt{rp\_deltamax($X$)}: $X$ is the \texttt{rprop} delta max.
\item \texttt{rp\_deltazero($X$)}: $X$ is the \texttt{rprop} delta zero.
\item \texttt{sp\_weight($X$)}:	$X$ is the \texttt{sarprop} weight decay.
\item \texttt{sp\_thresh($X$)}: $X$ is the \texttt{sarprop} step error threshold factor.
\item \texttt{sp\_shift($X$)}: $X$ is the \texttt{sarprop} step error shift.
\item \texttt{sp\_temp($X$)}: $X$ is the \texttt{sarprop} temperature.
\item \texttt{scale($InMin$,$InMax$,$OutMin$,$OutMax$)}: use training data scaling.
\item \texttt{inscale($InMin$,$InMax$)}:  use input data scaling.
\item \texttt{outscale($OutMin$,$OutMax$)}: use output data scaling.
\end{itemize}

\item \texttt{nn\_train($NN$,$Data$)}\index{\texttt{nn\_train/2}}: This predicate is the same as \texttt{nn\_train($NN$,$Data$,[])}.
\end{itemize}

\section{Save and Load Neural Networks}
\begin{itemize}
\item \texttt{nn\_save($NN$,$File$)}\index{\texttt{nn\_save/2}}: This predicate saves the neural network $NN$ to a file named $File$.

\item \texttt{nn\_load($File$) = $NN$}\index{\texttt{nn\_load/1}}: This function creates a neural network from a FANN neural network file named $File$.
\end{itemize}

\section{Run Neural Networks}
\begin{itemize}
\item \texttt{nn\_run($NN$,$Input$,$Opts$) = $Output$}\index{\texttt{nn\_run/3}}: This function predicts the output of a given input using the trained neural network $NN$, where $Input$ is an array of numbers or a single number, and $Opts$ is a list of testing options. The supported testing options include:
\begin{itemize}
\item \texttt{scaleIn($X$)}: $X$ indicates whether or not the input is scaled, with -1 meaning descale, 1 meaning scale, and 0 meaning nothing.
\item \texttt{scaleOut($X$)}: $X$ indicates whether or not the output is scaled,  with -1 meaning descale, 1 meaning scale, and 0 meaning nothing.	
\item \texttt{resetMSE}: resets the current mean squared error of the network.
\end{itemize}

\item \texttt{nn\_run($NN$,$Input$) = $Output$}\index{\texttt{nn\_run/2}}: This function is the same as \texttt{nn\_run($NN$,$Input$,[])}.
\end{itemize}

\ignore{
\end{document}
}

%\input{sockets.tex}
%\input{cinterface.tex}
\input{os.tex}
\input{math.tex}
\input{sys.tex}
\input{util.tex}
\input{ordset.tex}
\input{datetime.tex}
\input{format.tex}
\input{cinterface.tex}
\ignore{\input{prism.tex}}
%\input{sockopt.tex}
\input{lex_grammar.tex}
\input{syntax_grammar.tex}
\input{appendix_operators.tex}
\begin{adjustwidth}{-2cm}{-1cm}
\ignore{
\documentclass[10pt]{article}
\usepackage{fancyhdr}
\usepackage{graphicx}
\usepackage{wrapfig}
\usepackage{multicol}
\usepackage{setspace}
\usepackage{makeidx}
\newcommand{\ignore}[1]{}
\usepackage{vmargin}
\setpapersize{USletter}
\setmarginsrb{19mm}{19mm}{19mm}{19mm}{12pt}{5mm}{0pt}{0mm}
\usepackage{url}
\usepackage{times}
\usepackage{sectsty}
\sectionfont{\normalsize\bf}
\subsectionfont{\normalsize\bf}
\graphicspath{{figs/}}
\parindent 20pt
%\parskip 4pt
\newcommand{\mainfont}{ }
\newcommand{\headfont}{\itshape}
%
%\newcommand{\mainfont}{\sffamily}
%\pagestyle{fancy}
\lhead{{\bfseries\headfont Picat}}
%\rhead{{\bfseries\headfont Page~{\protect\thepage~of~\pageref{LastPage}}}}
\rhead{{\bfseries\headfont Page~{\protect\thepage}}}
%\chead{Google Faculty Research Awards Program}
\cfoot{}
\begin{document}
\begin{center}
\noindent {\bf The Library Modules of Picat} \\
\vspace*{5mm}
Neng-Fa Zhou and Jonathan Fruhman \\
\vspace*{5mm}
January 14, 2013 \\
\vspace*{5mm}
\end{center}
}
\chapter{Appendix: The Library Modules}
\setstretch{0.5}
\begin{multicols}{2}
\section*{ Module \texttt{basic} (imported by default)}
\begin{scriptsize}
\begin{itemize}
    \item {\tt $X$ \verb+!=+ $Y$}
    \item {\tt $X$ \verb+!==+ $Y$}
    \item {\tt $X$ \verb+:=+ $Y$}
    \item {\tt $X$ \verb+<+ $Y$}
    \item {\tt $X$ \verb+<=+ $Y$}
    \item {\tt $X$ \verb+=+ $Y$}
    \item {\tt $X$ \verb+=<+ $Y$}
    \item {\tt $X$ \verb+=:=+ $Y$}
    \item {\tt $X$ \verb+=\=+ $Y$}
    \item {\tt $X$ \verb+==+ $Y$}
    \item {\tt $X$ \verb+>+ $Y$}
    \item {\tt $X$ \verb+>=+ $Y$}
    \item {\tt $X$ @< $Y$}
    \item {\tt $X$ @<= $Y$}
    \item {\tt $X$ @=< $Y$}
    \item {\tt $X$ @> $Y$}
    \item {\tt $X$ @>= $Y$}
    \item {\tt $X$ =.. $Y$}
    \item {\tt $Term_1$ \verb-++- $Term_2$ = $List$}
    \item {\tt \verb+[+$X$ : $I$ in $D$,$\ldots$ \verb+]+ = $List$}
    \item {\tt $L$ \verb+..+ $U$ = $List$}
    \item {\tt $L$ \verb+..+ $Step$ \verb+..+ $U$ = $List$}
    \item {\tt \verb+-+$X$ = $Y$}
    \item {\tt \verb-+-$X$ = $Y$}
    \item {\tt $X$ \verb-+- $Y$ = $Z$}
    \item {\tt $X$ \verb+-+ $Y$ = $Z$}
    \item {\tt $X$ \verb+*+ $Y$ = $Z$}
    \item {\tt $X$ \verb+/+ $Y$ = $Z$}
    \item {\tt $X$ \verb+//+ $Y$ = $Z$}
    \item \texttt{$X$ div $Y$ = $Z$}
    \item {\tt $X$ \verb+/<+ $Y$ = $Z$}
    \item {\tt $X$ \verb+/>+ $Y$ = $Z$}
    \item {\tt $X$ \verb+**+ $Y$ = $Z$}
    \item \texttt{$X$ mod $Y$ = $Z$}
    \item \texttt{$X$ rem $Y$ = $Z$}
    \item {\tt \verb+~+$X$ = $Y$}
    \item {\tt $X$ \verb+\/+ $Y$ = $Z$}
    \item {\tt $X$ \verb+/\+ $Y$ = $Z$}
    \item {\tt $X$ \verb+^+ $Y$ = $Z$}
    \item {\tt $X$ \verb+<<+ $Y$ = $Z$}
    \item {\tt $X$ \verb+>>+ $Y$ = $Z$}
    \item {\tt $Var$\verb+[+$Index_1$,$\ldots$,$Index_n$\verb+]+}
    \item \texttt{$Goal_1$,$Goal_2$}
    \item \texttt{$Goal_1$ \&\& $Goal_2$}
    \item \texttt{$Goal_1$;$Goal_2$}
    \item \texttt{$Goal_1$ || $Goal_2$}
    \item \texttt{acyclic\_term($Term$)}
    \item \texttt{and\_to\_list($Conj$) = $List$}
    \item \texttt{append($X$,$Y$,$Z$)}  (nondet)
    \item \texttt{append($X$,$Y$,$Z$,$T$)}  (nondet)
    \item \texttt{apply($S$,$Arg_1$,$\ldots$,$Arg_n$) = $Val$}
    \item \texttt{arg($I$,$T$,$A$)}
    \item \texttt{arity($Struct$) = $Arity$}
    \item \texttt{array($Term$)}
    \item \texttt{ascii\_alpha($Term$)}
    \item \texttt{ascii\_alpha\_digit($Term$)}
    \item \texttt{ascii\_digit($Term$)}
    \item \texttt{ascii\_lowercase($Term$)}
    \item \texttt{ascii\_uppercase($Term$)}
    \item \texttt{atom($Term$)}
    \item \texttt{atom\_chars($Atm$) = $String$}
    \item \texttt{atom\_codes($Atm$) = $List$}
    \item \texttt{atomic($Term$)}
    \item \texttt{attr\_var($Term$)}
    \item \texttt{avg($ListOrArray$) = $Val$}
    \item \texttt{between($From$,$To$,$X$)} (nondet)
    \item \texttt{bigint($Term$)}
    \item \texttt{bind\_vars($Term$,$Val$)}
    \item \texttt{call($S$,$Arg_1$,$\ldots$,$Arg_n$)}
    \item \texttt{call\_cleanup($S$,$Cleanup$)}
    \item \texttt{catch($S$,$Exception$,$Handler$)}
    \item \texttt{char($Term$)}
    \item \texttt{chr($Code$) = $Char$}
    \item \texttt{clear($Map$)}
    \item \texttt{compare\_terms($Term_1$,$Term_2$) = $Res$}
    \item \texttt{compound($Term$)}
    \item \texttt{copy\_term($Term_1$) = $Term_2$}
    \item \texttt{count\_all($Call$) = $Int$}
    \item \texttt{delete($List$,$X$) = $ResList$}
    \item \texttt{delete\_all($List$,$X$) = $ResList$}
    \item \texttt{different\_terms($Term_1$,$Term_2$)}
    \item \texttt{digit($Char$)}
    \item \texttt{dvar($Term$)}
    \item \texttt{dvar\_or\_int($Term$)}
    \item \texttt{fail}
    \item \texttt{false}
    \item \texttt{find\_all($Template$,$Call$) = $List$}
    \item \texttt{findall($Template$,$Call$) = $List$}
    \item \texttt{first($Compound$) = $Term$}
    \item \texttt{flatten($List1$) = $List2$}
    \item \texttt{float($Term$)}
    \item \texttt{fold($F$,$ACC$,$List$) = $Res$}
    \item \texttt{freeze($X$,$Goal$)}
    \item \texttt{functor($T$,$F$,$N$)}
    \item \texttt{get($Map$,$Key$) = $Val$}
    \item \texttt{get($Map$,$Key$,$DefaultVal$)=$Val$}
    \item \texttt{get\_attr($AttrVar$,$Key$) = $Val$}
    \item \texttt{get\_attr($AttrVar$,$Key$,$DefaultVal$)=$Val$}
    \item \texttt{get\_global\_map($ID$) = $Map$}
    \item \texttt{get\_global\_map() = $Map$}
    \item \texttt{get\_heap\_map($ID$) = $Map$}
    \item \texttt{get\_heap\_map() = $Map$}
    \item \texttt{get\_table\_map($ID$) = $Map$}
    \item \texttt{get\_table\_map() = $Map$}
    \item \texttt{ground($Term$)}
    \item \texttt{handle\_exception($Term$,$Term$)}
    \item \texttt{has\_key($Map$,$Key$)}
    \item \texttt{hash\_code($Term$) = $Int$}
    \item \texttt{head($List$) = $Term$}
    \item \texttt{heap\_is\_empty($Heap$)}
    \item \texttt{heap\_pop($Heap$) = $Elm$}
    \item \texttt{heap\_push($Heap$,$Elm$)}
    \item \texttt{heap\_size($Heap$) = $Size$}
    \item \texttt{heap\_to\_list($Heap$) = $List$}
    \item \texttt{heap\_top($Heap$) = $Elm$}
    \item \texttt{insert($List$,$Index$,$Elm$) = $ResList$}
    \item \texttt{insert\_all($List$,$Index$,$AList$) = $ResList$}
    \item \texttt{insert\_ordered($List$,$Term$) = $R$}
    \item \texttt{insert\_ordered\_down($List$,$Term$) = $R$}
    \item \texttt{int($Term$)}
    \item \texttt{integer($Term$)}
    \item \texttt{is($T_1$,$T_2$)}
    \item \texttt{keys($Map$) = $List$}
    \item \texttt{last($Compound$) = $Term$}
    \item \texttt{len($Term$) = $Len$}
    \item \texttt{length($Term$) = $Len$}
    \item \texttt{list($Term$)}
    \item \texttt{list\_to\_and($List$) = $Conj$}
    \item \texttt{lowercase($Char$)}
    \item \texttt{map($Func$,$List1$,$List2$) = $ResList$}
    \item \texttt{map($FuncOrList$,$ListOrFunc$) = $ResList$}
    \item \texttt{map($Term$)}
    \item \texttt{map\_to\_list($Map$) = $List$}
    \item \texttt{max($ListOrArray$) = $Val$}
    \item \texttt{max($X$,$Y$) = $Val$}
    \item \texttt{maxint\_small() = $Int$}
    \item \texttt{maxof($Call$,$Objective$)}
    \item \texttt{maxof($Call$,$Objective$,$ReportCall$)}
    \item \texttt{maxof\_inc($Call$,$Objective$)}
    \item \texttt{maxof\_inc($Call$,$Objective$,$ReportCall$)}
    \item \texttt{membchk($Term$,$List$)}
    \item \texttt{member($Term$,$List$)} (nondet)
    \item \texttt{min($ListOrArray$) = $Val$}
    \item \texttt{min($X$,$Y$) = $Val$}
    \item \texttt{minint\_small() = $Int$}
    \item \texttt{minof($Call$,$Objective$)}
    \item \texttt{minof($Call$,$Objective$,$ReportCall$)}
    \item \texttt{minof\_inc($Call$,$Objective$)}
    \item \texttt{minof\_inc($Call$,$Objective$,$ReportCall$)}
    \item \texttt{name($Struct$) = $Name$}
    \item \texttt{new\_array($D_1$,$\ldots$,$D_n$) = $Arr$}
    \item \texttt{new\_list($N$) = $List$}
    \item \texttt{new\_list($N$,$InitVal$) = $List$}
    \item \texttt{new\_map($Int$,$PairsList$) = $Map$}
    \item \texttt{new\_map($IntOrPairsList$) = $Map$}
    \item \texttt{new\_max\_heap($IntOrList$) = $Heap$}
    \item \texttt{new\_min\_heap($IntOrList$) = $Heap$}
    \item \texttt{new\_set($Int$,$ElmsList$) = $Map$}
    \item \texttt{new\_set($IntOrElmsList$) = $Map$}
    \item \texttt{new\_struct($Name$,$IntOrList$) = $Struct$}
    \item \texttt{nonvar($Term$)}
    \item \texttt{not $Call$}
    \item \texttt{nth($I$,$ListOrArray$,$Val$)} (nondet)
    \item \texttt{number($Term$)}
    \item \texttt{number\_chars($Num$) = $String$}
    \item \texttt{number\_codes($Num$) = $List$}
    \item \texttt{number\_vars($Term$)}
    \item \texttt{number\_vars($Term$,$N_0$) = $N_1$}
    \item \texttt{once $Call$}
    \item \texttt{ord($Char$) = $Int$}
    \item \texttt{parse\_radix\_string($String$,$Base$) = $Int$}
    \item \texttt{parse\_term($String$) = $Term$}
    \item \texttt{parse\_term($String$,$Term$,$Vars$)}
    \item \texttt{post\_event($X$,$Event$)}
    \item \texttt{post\_event\_any($X$,$Event$)}
    \item \texttt{post\_event\_bound($X$)}
    \item \texttt{post\_event\_dom($X$,$Event$)}
    \item \texttt{post\_event\_ins($X$)}
    \item \texttt{prod($ListOrArray$) = $Val$}
    \item \texttt{put($Map$,$Key$)}
    \item \texttt{put($Map$,$Key$,$Val$)}
    \item \texttt{put\_attr($Var$,$Key$)}
    \item \texttt{put\_attr($Var$,$Key$,$Val$)}
    \item \texttt{real($Term$)}
    \item \texttt{reduce($Func$,$List$) = $Res$}
    \item \texttt{reduce($Func$,$List$,$InitVal$) = $Res$}
    \item \texttt{remove\_dups($ListOrArray$) = $ResList$}
    \item \texttt{repeat} (nondet)
    \item \texttt{reverse($ListOrArray$) = $Res$}
    \item \texttt{second($Compound$) = $Term$}
    \item \texttt{select($X$,$List$,$ResList$)} (nondet)
    \item \texttt{size($Map$) = $Size$}
    \item \texttt{slice($ListOrArray$,$From$)}
    \item \texttt{slice($ListOrArray$,$From$,$To$)}
    \item \texttt{sort($ListOrArray$) = $Sorted$}
    \item \texttt{sort($ListOrArray$,$KeyIndex$) = $Sorted$}
    \item \texttt{sort\_down($ListOrArray$) = $Sorted$}
    \item \texttt{sort\_down($ListOrArray$,$KeyIndex$) = $Sorted$}
    \item \texttt{sort\_down\_remove\_dups($ListOrArray$) = $Sorted$}
    \item \texttt{sort\_down\_remove\_dups($ListOrArray$,$KeyIndex$) = $Sorted$}
    \item \texttt{sort\_remove\_dups($ListOrArray$) = $Sorted$}
    \item \texttt{sort\_remove\_dups($ListOrArray$,$KeyIndex$) = $Sorted$}
    \item \texttt{sorted($ListOrArray$)}
    \item \texttt{sorted\_down($ListOrArray$)}
    \item \texttt{string($Term$)}
    \item \texttt{struct($Term$)}
    \item \texttt{subsumes($Term_1$,$Term_2$)}
    \item \texttt{sum($ListOrArray$) = $Val$}
    \item \texttt{tail($List$) = $Term$}
    \item \texttt{throw($E$)}
    \item \texttt{to\_array($List$) = $Array$}
    \item \texttt{to\_atom($String$) = $Atom$}
    \item \texttt{to\_binary\_string($Int$) = $String$}
    \item \texttt{to\_codes($Term$) = $List$}
    \item \texttt{to\_fstring($Format$,$Args\ldots$) = $String$}
    \item \texttt{to\_hex\_string($Int$) = $String$}
    \item \texttt{to\_int($NumOrCharOrStr$) = $Int$}
    \item \texttt{to\_integer($NumOrCharOrStr$) = $Int$}
    \item \texttt{to\_list($Struct$) = $List$}
    \item \texttt{to\_lowercase($String$) = $LString$}
    \item \texttt{to\_number($NumOrCharOrStr$) = $Number$}
    \item \texttt{to\_oct\_string($Int$) = $String$}
    \item \texttt{to\_radix\_string($Int$,$Base$) = $String$}
    \item \texttt{to\_real($NumOrStr$) = $Real$}
    \item \texttt{to\_string($Term$) = $String$}
    \item \texttt{to\_uppercase($String$) = $UString$}
    \item \texttt{true}
    \item \texttt{uppercase($Char$)}
    \item \texttt{values($Map$) = $List$}
    \item \texttt{var($Term$)}
    \item \texttt{variant($Term_1$,$Term_2$)}
    \item \texttt{vars($Term$) = $Vars$}
    \item \texttt{zip($List_1$,$List_2$) = $List$}
    \item \texttt{zip($List_1$,$List_2$,$List_3$) = $List$}
    \item \texttt{zip($List_1$,$List_2$,$List_3$,$List_4$) = $List$}
%    \item \texttt{unnumber\_vars($Term_1$) = $Term_2$}
\end{itemize}
\end{scriptsize}
%
\section*{Module \texttt{math} (imported by default)}
\begin{scriptsize}
\begin{itemize}
   \item \texttt{abs($X$) = $Val$}
   \item \texttt{acos($X$) = $Val$}
   \item \texttt{acosh($X$) = $Val$}
   \item \texttt{acot($X$) = $Val$}
   \item \texttt{acoth($X$) = $Val$}
   \item \texttt{acsc($X$) = $Val$}
   \item \texttt{acsch($X$) = $Val$}
   \item \texttt{asec($X$) = $Val$}
   \item \texttt{asech($X$) = $Val$}
   \item \texttt{asin($X$) = $Val$}
   \item \texttt{asinh($X$) = $Val$}
   \item \texttt{atan($X$) = $Val$}
   \item \texttt{atan2($X$,$Y$) = $Val$}
   \item \texttt{atanh($X$) = $Val$}
   \item \texttt{ceiling($X$) = $Val$}
   \item \texttt{cos($X$) = $Val$}
   \item \texttt{cosh($X$)  = $Val$}
   \item \texttt{cot($X$) = $Val$}
   \item \texttt{coth($X$)  = $Val$}
   \item \texttt{csc($X$) = $Val$}
   \item \texttt{csch($X$)  = $Val$}
   \item \texttt{e() = 2.71828182845904523536}
   \item \texttt{even($Int$)}
   \item \texttt{exp($X$) = $Val$}
   \item \texttt{floor($X$) = $Val$}
   \item \texttt{frand() = $Val$}
   \item \texttt{frand($Low$,$High$) = $Val$}
   \item \texttt{gcd($A$,$B$) = $Val$}
   \item \texttt{log($X$) = $Val$}
   \item \texttt{log(B,$X$) = $Val$}
   \item \texttt{log10($X$) = $Val$}
   \item \texttt{log2($X$) = $Val$}
   \item \texttt{modf($X$) = ($IntVal$,$FractVal$)}
   \item \texttt{odd($Int$)}
   \item \texttt{pi() = 3.14159265358979323846}
   \item \texttt{pow($X$,$Y$) = $Val$}
   \item \texttt{pow\_mod($X$,$Y$,$Z$) = $Val$}
   \item \texttt{prime($Int$)}
   \item \texttt{primes($Int$) = $List$}
   \item \texttt{rand\_max() = $Val$}
   \item \texttt{random = $Val$}
   \item \texttt{random($Low$,$High$) = $Val$}
   \item \texttt{random($Seed$) = $Val$}
   \item \texttt{random2() = $Int$}
   \item \texttt{round($X$) = $Val$}
   \item \texttt{sec($X$) = $Val$}
   \item \texttt{sech($X$) = $Val$}
   \item \texttt{sign($X$) = $Val$}
   \item \texttt{sin($X$) = $Val$}
   \item \texttt{sinh($X$) = $Val$}
   \item \texttt{sqrt($X$) = $Val$}
   \item \texttt{tan($X$) = $Val$}
   \item \texttt{tanh($X$) = $Val$}
   \item \texttt{to\_degrees($Radian$) = $Degree$}
   \item \texttt{to\_radians($Degree$) = $Radian$}
   \item \texttt{truncate($X$) = $Val$}
%   \item \texttt{inf}
%   \item \texttt{acosh($X$) = $Val$}
%   \item \texttt{asinh($X$) = $Val$}
%   \item \texttt{atanh($X$) = $Val$}
%   \item \texttt{cbrt($X$) = $Val$}
%   \item \texttt{cosh($X$) = $Val$}
%   \item \texttt{coth($X$) = $Val$}
%   \item \texttt{ninf}
%   \item \texttt{nthrt($N$,$X$) = $Val$}
%   \item \texttt{randrange($From$,$Step$,$To$) = $Val$}
%   \item \texttt{randrange($From$,$To$) = $Val$}
%   \item \texttt{sech($X$) = $Val$}
%   \item \texttt{sinh($X$) = $Val$}
%   \item \texttt{tanh($X$) = $Val$}
%   \item \texttt{csch($X$) = $Val$}
\end{itemize}
\end{scriptsize}
\section*{Module \texttt{io} (imported by default)}
\begin{scriptsize}
\begin{itemize}
   \item \texttt{at\_end\_of\_stream($FD$)}
   \item \texttt{close($FD$)}
   \item \texttt{flush($FD$)}
   \item \texttt{flush()}
   \item \texttt{nl($FD$)}
   \item \texttt{nl()}
   \item \texttt{open($Name$) = $FD$}
   \item \texttt{open($Name$,$Mode$) = $FD$}
   \item \texttt{peek\_byte($FD$) = $Val$}
   \item \texttt{peek\_char($FD$) = $Val$}
   \item \texttt{print($FD$,$Term$)}
   \item \texttt{print($Term$)}
   \item \texttt{printf($FD$,$Format$,$Args\ldots$)}
   \item \texttt{println($FD$,$Term$)}
   \item \texttt{println($Term$)}
   \item \texttt{read\_atom($FD$) = $Atom$}
   \item \texttt{read\_atom() = $Atom$}
   \item \texttt{read\_byte($FD$) = $Val$}
   \item \texttt{read\_byte($FD$,$N$) = $List$}
   \item \texttt{read\_byte() = $Val$}
   \item \texttt{read\_char($FD$) = $Val$}
   \item \texttt{read\_char($FD$,$N$) = $String$}
   \item \texttt{read\_char() = $Val$}
   \item \texttt{read\_char\_code($FD$) = $Val$}
   \item \texttt{read\_char\_code($FD$,$N$) = $List$}
   \item \texttt{read\_char\_code() = $Val$}
   \item \texttt{read\_file\_bytes($File$) = $List$}
   \item \texttt{read\_file\_bytes() = $List$}
   \item \texttt{read\_file\_chars($File$) = $String$}
   \item \texttt{read\_file\_chars() = $String$}
   \item \texttt{read\_file\_codes($File$) = $List$}
   \item \texttt{read\_file\_codes() = $List$}
   \item \texttt{read\_file\_lines($File$) = $List$}
   \item \texttt{read\_file\_lines() = $List$}
   \item \texttt{read\_file\_terms($File$) = $List$}
   \item \texttt{read\_file\_terms() = $List$}
   \item \texttt{read\_int($FD$) = $Int$}
   \item \texttt{read\_int() = $Int$}
   \item \texttt{read\_line($FD$) = $String$}
   \item \texttt{read\_line() = $String$}
   \item \texttt{read\_number($FD$) = $Number$}
   \item \texttt{read\_number() = $Number$}
   \item \texttt{read\_picat\_token($FD$) = $TokenValue$}
   \item \texttt{read\_picat\_token($FD$,$TokenType$,$TokenValue$)}
   \item \texttt{read\_picat\_token($TokenType$,$TokenValue$)}
   \item \texttt{read\_picat\_token() = $TokenValue$}
   \item \texttt{read\_real($FD$) = $Real$}
   \item \texttt{read\_real() = $Real$}
   \item \texttt{read\_term($FD$) = $Term$}
   \item \texttt{read\_term() = $Term$}
   \item \texttt{readln($FD$) = $String$}
   \item \texttt{readln() = $String$}
   \item \texttt{write($FD$,$Term$)}
   \item \texttt{write($Term$)}
   \item \texttt{write\_byte($Bytes$)}
   \item \texttt{write\_byte($FD$,$Bytes$)}
   \item \texttt{write\_char($Chars$)}
   \item \texttt{write\_char($FD$,$Chars$)}
   \item \texttt{write\_char\_code($Codes$)}
   \item \texttt{write\_char\_code($FD$,$Codes$)}
   \item \texttt{writef($FD$,$Format$,$Args\ldots$)}
   \item \texttt{writeln($FD$,$Term$)}
   \item \texttt{writeln($Term$)}
\ignore{
%   \item \texttt{dup($FD$) = $NewFD$}
%   \item \texttt{dup2($FromFD$,$ToFD$)}
%   \item \texttt{eof}
   \item \texttt{peek\_int($FD$) = $Int$}
   \item \texttt{peek\_real($FD$) = $Real$}
%   \item \texttt{read\_unicode\_char($FD$) = $Val$}
%   \item \texttt{read\_unicode\_char($FD$,$N$) = $String$}
   \item \texttt{getpos($FD$) = $Pos$}
   \item \texttt{mkfifo($Path$)}
   \item \texttt{mkfifo($Path$,$Mode$)}
   \item \texttt{mkpipe() = $FD\_Map$}
   \item \texttt{mktmp() = $FD$}
   \item \texttt{peek\_unicode\_char($FD$) = $Val$}
   \item \texttt{rewind($FD$)}
   \item \texttt{seek($FD$,$Offset$,$From$)}
   \item \texttt{setpos($FD$,$Pos$)}
   \item \texttt{sizeof\_char() = $Size$}
   \item \texttt{stderr}
   \item \texttt{stdin}
   \item \texttt{stdout}
}
\end{itemize}
\end{scriptsize}
\section*{Module \texttt{ordset}}
\begin{scriptsize}
\begin{itemize}
\item \texttt{delete($OSet$,$Elm$) = $OSet1$}
\item \texttt{disjoint($OSet1$,$OSet2$)}
\item \texttt{insert($OSet$,$Elm$) = $OSet1$}
\item \texttt{intersection($OSet1$,$OSet2$)=$OSet3$}
\item \texttt{new\_ordset($List$)}
\item \texttt{ordset($Term$)}
\item \texttt{subset($OSet1$,$OSet2$)}
\item \texttt{subtract($OSet1$,$OSet2$)=$OSet3$}
\item \texttt{union($OSet1$,$OSet2$)=$OSet3$}
\end{itemize}
\end{scriptsize}
\section*{Module \texttt{os}}
\begin{scriptsize}
\begin{itemize}
\item \texttt{cd($Path$)}
\item \texttt{chdir($Path$)}
\item \texttt{cp($FromPath$,$ToPath$)}
\item \texttt{cwd() = $Path$}
\item \texttt{directory($Path$)}
\item \texttt{dir}
\item \texttt{env\_exists($Name$)}
\item \texttt{executable($Path$)}
\item \texttt{exists($Path$)}
\item \texttt{file($Path$)}
\item \texttt{file\_base\_name($Path$) = $String$}
\item \texttt{file\_directory\_name($Path$) = $String$}
\item \texttt{file\_exists($Path$)}
\item \texttt{getenv($EnvString$) = $String$}
\item \texttt{listdir($Path$) = $List$}
\item \texttt{ls}
\item \texttt{mkdir($Path$)}
\item \texttt{pwd() = $Path$}
\item \texttt{readable($Path$)}
\item \texttt{rename($Old$,$New$)}
\item \texttt{rm($Path$)}
\item \texttt{rmdir($Path$)}
\item \texttt{separator() = $Val$}
\item \texttt{size($Path$) = $Int$}
\item \texttt{writable($Path$)}
\ignore{
\item \texttt{atime($Path$) = $DateTime$}
\item \texttt{block\_special($Path$)}
\item \texttt{char\_special($Path$)}
\item \texttt{chmod($Path$,$Mode$)}
\item \texttt{create($Path$)}
\item \texttt{create($Path$,$Mode$)}
\item \texttt{ctime($Path$) = $DateTime$}
\item \texttt{dev\_id($Path$) = $Int$}
\item \texttt{directory\_exists($Path$)}
\item \texttt{fifo($Path$)}
\item \texttt{file\_type($Path$) = $Term$}
\item \texttt{gid($Path$) = $Int$}
\item \texttt{ino($Path$) = $Int$}
\item \texttt{link($Path$)}
\item \texttt{link($Path1$,$Path2$)}
\item \texttt{listdir($Path$,$REPattern$) = $List$}
\item \texttt{message\_queue($Path$)}
\item \texttt{mkdir($Path$,$Mode$)}
\item \texttt{mkdirs($Path$)}
\item \texttt{mkdirs($Path$,$Mode$)}
\item \texttt{mode($Path$) = $String$}
\item \texttt{mode($Path$,$Value$)}
\item \texttt{mtime($Path$) = $DateTime$}
\item \texttt{mv($Path1$,$Path2$)}
\item \texttt{nlink($Path$) = $Int$}
\item \texttt{root() = $Path$}
\item \texttt{semaphore($Path$)}
\item \texttt{shared\_memory($Path$)}
\item \texttt{shortcut($Path$)}
\item \texttt{shortcut($Path1$,$Path2$)}
\item \texttt{socket($Path$)}
\item \texttt{uid($Path$) = $Int$}
\item \texttt{unlink($Path$)}
}
\end{itemize}
\end{scriptsize}
\section*{Modules \texttt{cp}, \texttt{sat}, \texttt{smt}, and \texttt{mip}}
\begin{scriptsize}
\begin{itemize}
    \item {\tt \verb+#~+$X$}
    \item {\tt $X$ \verb+#!=+ $Y$}
    \item {\tt $X$ \verb+#/\+ $Y$}
    \item {\tt $X$ \verb+#<+ $Y$}
    \item {\tt $X$ \verb+#<=+ $Y$}
    \item {\tt $X$ \verb+#<=>+ $Y$}
    \item {\tt $X$ \verb+#=+ $Y$}
    \item {\tt $X$ \verb+#=<+ $Y$}
    \item {\tt $X$ \verb+#=>+ $Y$}
    \item {\tt $X$ \verb+#>+ $Y$}
    \item {\tt $X$ \verb+#>=+ $Y$}
    \item {\tt $X$ \verb+#\/+ $Y$}
    \item {\tt $X$ \verb+#^+ $Y$}
\item \texttt{$Vars$ :: $Exp$}
\item \texttt{$Vars$ notin $Exp$}
\item \texttt{all\_different($FDVars$)}
\item \texttt{all\_different\_except\_0($FDVars$)}
\item \texttt{all\_distinct($FDVars$)}
\item \texttt{assignment($FDVars1$,$FDVars2$)}
\item \texttt{at\_least($N$,$L$,$V$)}:
\item \texttt{at\_most($N$,$L$,$V$)}:
\item \texttt{circuit($FDVars$)}
\item \texttt{count($V$,$FDVars$,$N$)}
\item \texttt{count($V$,$FDVars$,$Rel$,$N$)}
\item \texttt{cumulative($Ss$,$Ds$,$Rs$,$Limit$)}
\item \texttt{decreasing($L$)}
\item \texttt{decreasing\_strict($L$)}
\item \texttt{diffn($RectangleList$)}
\item \texttt{disjunctive\_tasks($Tasks$)} (cp only)
\item \texttt{element($I$,$List$,$V$)}
\item \texttt{exactly($N$,$L$,$V$)}:
\item \texttt{fd\_degree($FDVar$) = $Degree$} (cp only)
\item \texttt{fd\_disjoint($DVar1$,$DVar2$)}
\item \texttt{fd\_dom($FDVar$) = $List$}
\item \texttt{fd\_false($FDVar$,$Elm$)}
\item \texttt{fd\_max($FDVar$) = $Max$}
\item \texttt{fd\_min($FDVar$) = $Min$}
\item \texttt{fd\_min\_max($FDVar$,$Min$,$Max$)}
\item \texttt{fd\_next($FDVar$,$Elm$) = $NextElm$}
\item \texttt{fd\_prev($FDVar$,$Elm$) = $PrevElm$}
\item \texttt{fd\_set\_false($FDVar$,$Elm$)} (cp only)
\item \texttt{fd\_size($FDVar$) = $Size$}
\item \texttt{fd\_true($FDVar$,$Elm$)}
\item \texttt{fd\_vector\_min\_max($Min$,$Max$)}
\item \texttt{global\_cardinality($List$,$Pairs$)}
\item \texttt{increasing($L$)}
\item \texttt{increasing\_strict($L$)}
\item \texttt{indomain($Var$)} (nondet) (cp only)
\item \texttt{indomain\_down($Var$)} (nondet) (cp only)
\item \texttt{lex\_le($L_1$,$L_2$)}
\item \texttt{lex\_lt($L_1$,$L_2$)}
\item \texttt{matrix\_element($Matrix$,$I$,$J$,$V$)}
\item \texttt{neqs($NeqList$)} (cp only)
\item \texttt{new\_dvar() = $FDVar$}
\item \texttt{new\_fd\_var() = $FDVar$}
\item \texttt{regular$(X,Q,S,D,Q0,F)$}
\item \texttt{scalar\_product($A$,$X$,$Product$)}
\item \texttt{scalar\_product($A$,$X$,$Rel$,$Product$)}
\item \texttt{serialized($Starts$,$Durations$)}
\item \texttt{solve($Opts$,$Vars$)} (nondet)
\item \texttt{solve($Vars$)} (nondet)
\item \texttt{solve\_all($Opts$,$Vars$) = $List$}
\item \texttt{solve\_all($Vars$) = $List$}
\item \texttt{solve\_suspended} (cp only)
\item \texttt{solve\_suspended($Opt$)} (cp only)
\item \texttt{subcircuit($FDVars$)}
\item \texttt{subcircuit\_grid($A$) (sat only)}
\item \texttt{subcircuit\_grid($A$,$K$) (sat only)}
\item \texttt{table\_in($DVars$,$R$)}
\item \texttt{table\_notin($DVars$,$R$)}
\end{itemize}
\end{scriptsize}
\section*{Module \texttt{planner}}
\begin{scriptsize}
\begin{itemize}
\item \texttt{best\_plan($S$,$Limit$,$Plan$)}
\item \texttt{best\_plan($S$,$Limit$,$Plan$,$Cost$)}
\item \texttt{best\_plan($S$,$Plan$)}
\item \texttt{best\_plan($S$,$Plan$,$PlanCost$)}
\item \texttt{best\_plan\_bb($S$,$Limit$,$Plan$)}
\item \texttt{best\_plan\_bb($S$,$Limit$,$Plan$,$Cost$)}
\item \texttt{best\_plan\_bb($S$,$Plan$)}
\item \texttt{best\_plan\_bb($S$,$Plan$,$PlanCost$)}
\item \texttt{best\_plan\_bin($S$,$Limit$,$Plan$)}
\item \texttt{best\_plan\_bin($S$,$Limit$,$Plan$,$Cost$)}
\item \texttt{best\_plan\_bin($S$,$Plan$)}
\item \texttt{best\_plan\_bin($S$,$Plan$,$PlanCost$)}
\item \texttt{best\_plan\_nondet($S$,$Limit$,$Plan$)} (nondet)
\item \texttt{best\_plan\_nondet($S$,$Limit$,$Plan$,$Cost$)} (nondet)
\item \texttt{best\_plan\_nondet($S$,$Plan$)} (nondet)
\item \texttt{best\_plan\_nondet($S$,$Plan$,$PlanCost$)} (nondet)
\item \texttt{best\_plan\_unbounded($S$,$Limit$,$Plan$)}
\item \texttt{best\_plan\_unbounded($S$,$Limit$,$Plan$,$Cost$)}
\item \texttt{best\_plan\_unbounded($S$,$Plan$)}
\item \texttt{best\_plan\_unbounded($S$,$Plan$,$PlanCost$)}
\item \texttt{current\_plan()=$Plan$}
\item \texttt{current\_resource()=$Amount$}
\item \texttt{current\_resource\_plan\_cost($Amount$,$Plan$,$Cost$)}
\item \texttt{is\_tabled\_state($S$)}
\item \texttt{plan($S$,$Limit$,$Plan$)}
\item \texttt{plan($S$,$Limit$,$Plan$,$Cost$)}
\item \texttt{plan($S$,$Plan$)}
\item \texttt{plan($S$,$Plan$,$PlanCost$)}
\item \texttt{plan\_unbounded($S$,$Limit$,$Plan$)}
\item \texttt{plan\_unbounded($S$,$Limit$,$Plan$,$Cost$)}
\item \texttt{plan\_unbounded($S$,$Plan$)}
\item \texttt{plan\_unbounded($S$,$Plan$,$PlanCost$)}
\end{itemize}
\end{scriptsize}
\section*{Module \texttt{nn} (Neural Networks)}
\begin{scriptsize}
\begin{itemize}
\item \texttt{new\_nn($Layers$) = $NN$}
\item \texttt{new\_sparse\_nn($Layers$) = $NN$}
\item \texttt{new\_sparse\_nn($Layers$,$Rate$) = $NN$}
\item \texttt{new\_standard\_nn($Layers$) = $NN$}
\item \texttt{nn\_destroy($NN$)}
\item \texttt{nn\_destroy\_all}
\item \texttt{nn\_load($File$) = $NN$}
\item \texttt{nn\_print($NN$)}
\item \texttt{nn\_run($NN$,$Input$) = $Output$}
\item \texttt{nn\_run($NN$,$Input$,$Opts$) = $Output$}
\item \texttt{nn\_save($NN$,$File$)}
\item \texttt{nn\_set\_activation\_function\_hidden($NN$,$Func$)}
\item \texttt{nn\_set\_activation\_function\_layer($NN$,$Func$,$Layer$)}
\item \texttt{nn\_set\_activation\_function\_output($NN$,$Func$)}
\item \texttt{nn\_set\_activation\_steepness\_hidden($NN$,$Steepness$)}
\item \texttt{nn\_set\_activation\_steepness\_layer($NN$,$Steepness$,$Layer$)}
\item \texttt{nn\_set\_activation\_steepness\_output($NN$,$Steepness$)}
\item \texttt{nn\_train($NN$,$Data$)}
\item \texttt{nn\_train($NN$,$Data$,$Opts$)}
\item \texttt{nn\_train\_data\_get($Data$,$I$) = $Pair$}
\item \texttt{nn\_train\_data\_load($File$) = $Data$}
\item \texttt{nn\_train\_data\_save($Data$,$File$)}
\item \texttt{nn\_train\_data\_size($Data$) = $Size$}
\end{itemize}
\end{scriptsize}
\section*{Module \texttt{datetime}}
\begin{scriptsize}
\begin{itemize}
\item \texttt{current\_datetime() = $DateTime$}
\item \texttt{current\_day() = $WDay$}
\item \texttt{current\_date() = $Date$}
\item \texttt{current\_time() = $Time$}
\end{itemize}
\end{scriptsize}
\ignore{
\section*{Module \texttt{thread}}
\begin{scriptsize}
\begin{itemize}
    \item \texttt{acquire\_mutex($Mutex$)}
    \item \texttt{broadcast\_cv($CV$)}
    \item \texttt{join($Thread$)}
    \item \texttt{new\_cv() = $CV$}
    \item \texttt{new\_mutex() = $Mutex$}
    \item \texttt{new\_rwlock() = $RWLock$}
    \item \texttt{new\_semaphore() = $Semaphore$}
    \item \texttt{new\_semaphore($N$) = $Semaphore$}
    \item \texttt{new\_thread($S$,$Arg_1$,$\ldots$,$Arg_n$) = $Thread$}
    \item \texttt{p\_semaphore($Semaphore$)}
    \item \texttt{rdlock($RWLock$)}
    \item \texttt{release\_mutex($Mutex$)}
    \item \texttt{rwunlock($RWLock$)}
    \item \texttt{signal\_cv($CV$)}
    \item \texttt{sleep($Milliseconds$)}
    \item \texttt{start($Thread$)}
    \item \texttt{this\_thread() = $Thread$}
    \item \texttt{v\_semaphore($Semaphore$)}
    \item \texttt{wait\_cv($CV$,$Mutex$)}
    \item \texttt{wrlock($RWLock$)}
\end{itemize}
\end{scriptsize}
\section*{Module \texttt{timer}}
\begin{scriptsize}
\begin{itemize}
    \item \texttt{get\_interval($Timer$) = $Milliseconds$}
    \item \texttt{kill($Timer$)}
    \item \texttt{new\_timer($Milliseconds$) = $Timer$}
    \item \texttt{set\_interval($Timer$,$Milliseconds$)}
    \item \texttt{start($Timer$)}
    \item \texttt{stop($Timer$)}
\end{itemize}
\end{scriptsize}
\section*{Module \texttt{process}}
\begin{scriptsize}
\begin{itemize}
    \item \texttt{exec($S$,$Arg_1$,$\ldots$,$Arg_n$)}
    \item \texttt{execl($S$,$ArgList$)}
    \item \texttt{fork() = $ID$}
    \item \texttt{new\_process($S$,$Arg_1$,$\ldots$,$Arg_n$) = $ID$}
    \item \texttt{pid() = $ID$}
    \item \texttt{ppid() = $ID$}
    \item \texttt{wait() = $StatMap$}
    \item \texttt{waitpid($ID$) = $StatMap$}
\end{itemize}
\end{scriptsize}
\section*{Module \texttt{socket}}
\begin{scriptsize}
\begin{itemize}
    \item \texttt{accept($FD$) = $Client$}
    \item \texttt{bind($FD$,$INet$,$Address$,$Port$)}
    \item \texttt{bind($FD$,$Unix$,$Name$)}
    \item \texttt{close($FD$)}
    \item \texttt{connect($FD$,$INet$,$Address$,$Port$)}
    \item \texttt{connect($FD$,$Unix$,$Name$)}
    \item \texttt{getaddr($Name$) = $Addr$}
    \item \texttt{getcanonicalname($Addr$) = $Name$}
    \item \texttt{gethostbyaddr($Addr$) = $Host$}
    \item \texttt{gethostbyname($Name$) = $Host$}
    \item \texttt{getservbyname($Name$) = $Service$}
    \item \texttt{getservbyname($Name$,$Type$) = $Service$}
    \item \texttt{getservport($Name$) = $Port$}
    \item \texttt{getsockopt($FD$,$Level$,$Option$) = $Value$}
    \item \texttt{joingroup($GroupAddress$)}
    \item \texttt{leavegroup($GroupAddress$)}
    \item \texttt{listen($FD$)}
    \item \texttt{listen($FD$,$Backlog$)}
    \item \texttt{recv($FD$) = $Message$}
    \item \texttt{recv($FD$,$Flags$) = $Message$}
    \item \texttt{recvfrom($FD$,$Domain$) = $Message$}
    \item \texttt{recvfrom($FD$,$Flags$,$Domain$) = $Message$}
    \item \texttt{send($FD$,$Message$) = $NBytes$}
    \item \texttt{send($FD$,$Message$,$Flags$) = $NBytes$}
    \item \texttt{sendto($FD$,$Message$,$Domain$,$Address$,$Port$) = $NBytes$}
    \item \texttt{sendto($FD$,$Message$,$Flags$,\\        $Domain$,$Address$,$Port$) = $NBytes$}
    \item \texttt{sendto($FD$,$Message$,$Flags$,$Name$) = $NBytes$}
    \item \texttt{sendto($FD$,$Message$,$Name$) = $NBytes$}
    \item \texttt{setsockopt($FD$,$Level$,$Option$,$Value$)}
    \item \texttt{socket($Domain$,$Type$) = $FD$}
    \item \texttt{tcp\_bind($FD$,$Address$,$Port$)}
    \item \texttt{tcp\_connect($FD$,$Address$,$Port$)}
    \item \texttt{tcp\_socket() = $FD$}
    \item \texttt{udp\_bind($FD$,$Address$,$Port$)}
    \item \texttt{udp\_socket() = $FD$}
    \item \texttt{unix\_bind($FD$,$Name$)}
    \item \texttt{unix\_connect($FD$,$Name$)}
    \item \texttt{unix\_socket() = $FD$}
\end{itemize}
\end{scriptsize}
}
\section*{Module \texttt{sys} (imported by default)}
\begin{scriptsize}
\begin{itemize}
    \item \texttt{abort}
    \item \texttt{cl}
    \item \texttt{cl($File$)}
    \item \texttt{cl\_facts($Facts$)}
    \item \texttt{cl\_facts($Facts$,$IndexInfo$)}
    \item \texttt{cl\_facts\_table($Facts$)}
    \item \texttt{cl\_facts\_table($Facts$,$IndexInfo$)}
    \item \texttt{command($String$)}
    \item \texttt{compile($File$)}
    \item \texttt{debug}
    \item \texttt{exit}
    \item \texttt{garbage\_collect}
    \item \texttt{garbage\_collect(Size)}
    \item \texttt{halt}
    \item \texttt{initialize\_table}
    \item \texttt{load($File$)}
    \item \texttt{loaded\_modules()}
    \item \texttt{nodebug}
    \item \texttt{nospy}
    \item \texttt{notrace}
    \item \texttt{spy $Functor$}
    \item \texttt{statistics}
    \item \texttt{statistics($Name$,$Value$)} (nondet)
    \item \texttt{statistics\_all() = $List$}
    \item \texttt{time($Goal$)}
    \item \texttt{time2($Goal$)}
    \item \texttt{time\_out($Goal$,$Limit$,$Res$)}
    \item \texttt{trace}
%    \item \texttt{execute($CommandString$) = $Status$}
%    \item \texttt{exit}
%    \item \texttt{help}
%    \item \texttt{modules() = $List$}
%    \item \texttt{profile($Goal$)}
%    \item \texttt{profile\_src($File$)}
%    \item \texttt{prompt($NewPrompt$)}
%    \item \texttt{table\_get\_all($Goal$) = $List$}
%    \item \texttt{table\_get\_one($Goal$)}
\end{itemize}
\end{scriptsize}
\section*{Module \texttt{util}}
\begin{scriptsize}
\begin{itemize}
%    \item \texttt{ = }
    \item \texttt{array\_matrix\_to\_list($Matrix$) = $List$}
    \item \texttt{array\_matrix\_to\_list\_matrix($AMatrix$) = $LMatrix$}
    \item \texttt{chunks\_of($List$,$K$) = $ListOfLists$}
    \item \texttt{find($String$,$SubString$,$From$,$To$)} (nondet)
    \item \texttt{find\_first\_of($Term$,$Pattern$) = $Index$}
    \item \texttt{find\_ignore\_case($String$,$SubString$,$From$,$To$)} (nondet)
    \item \texttt{find\_last\_of($Term$,$Pattern$) = $Index$}
    \item \texttt{join($Words$) = $String$}
    \item \texttt{join($Words$,$Separator$) = $String$}
    \item \texttt{list\_matrix\_to\_array\_matrix($LMatrix$) = $AMatrix$}
    \item \texttt{lstrip($List$) = $List$ }
    \item \texttt{lstrip($List$,$Elms$) = $List$ }
    \item \texttt{matrix\_multi($MatrixA$,$MatrixB$) = $MatrixC$}
    \item \texttt{permutation($List$,$Perm$)} (nondet)
    \item \texttt{permutations($List$) = $Lists$}
    \item \texttt{power\_set($List$) = $Lists$}
    \item \texttt{replace($Term$,$Old$,$New$) = $NewTerm$}
    \item \texttt{replace\_at($Term$,$Index$,$New$) = $NewTerm$}
    \item \texttt{rstrip($List$) = $List$ }
    \item \texttt{rstrip($List$,$Elms$) = $List$ }
    \item \texttt{split($List$) = $Words$ }
    \item \texttt{split($List$,$Separators$) = $Words$ }
    \item \texttt{strip($List$) = $List$ }
    \item \texttt{strip($List$,$Elms$) = $List$ }
    \item \texttt{take($List$,$K$) = $List$}
    \item \texttt{transpose($Matrix$) = $Transposed$}
\end{itemize}
\end{scriptsize}
\ignore{
\section*{Module \texttt{datetime}}
\begin{scriptsize}
\begin{itemize}
    \item \texttt{add\_days($DateTime$,$Days$) = $DateTime$}
    \item \texttt{add\_hours($DateTime$,$Hours$) = $DateTime$}
    \item \texttt{add\_milliseconds($DateTime$,$MilliSeconds$) = $DateTime$}
    \item \texttt{add\_minutes($DateTime$,$Minutes$) = $DateTime$}
    \item \texttt{add\_months($DateTime$,$Months$) = $DateTime$}
    \item \texttt{add\_seconds($DateTime$,$Seconds$) = $DateTime$}
    \item \texttt{add\_years($DateTime$,$Years$) = $DateTime$}
    \item \texttt{compare($DateTime$,$DateTime$) = $Res$}
    \item \texttt{current\_datetime() = $DateTime$}
    \item \texttt{day($DateTime$) = $Day$}
    \item \texttt{day\_of\_week($DateTime$) = $Atom$}
    \item \texttt{day\_of\_year($DateTime$) = $Int$}
    \item \texttt{day\_string($DateTime$) = $String$}
    \item \texttt{dt\_to\_fstring($Format$,$DateTime$) = $String$}
    \item \texttt{hour($DateTime$) = $Hour$}
    \item \texttt{is\_leap\_year($DateTime$)}
    \item \texttt{millisecond($DateTime$) = $MilliSecond$}
    \item \texttt{minute($DateTime$) = $Minute$}
    \item \texttt{month($DateTime$) = $Month$}
    \item \texttt{month\_string($DateTime$) = $String$}
    \item \texttt{second($DateTime$) = $Second$}
    \item \texttt{set\_day($DateTime$,$Day$)}
    \item \texttt{set\_hour($DateTime$,$Hour$)}
    \item \texttt{set\_millisecond($DateTime$,$MilliSecond$)}
    \item \texttt{set\_minute($DateTime$,$Minute$)}
    \item \texttt{set\_month($DateTime$,$Month$)}
    \item \texttt{set\_second($DateTime$,$Second$)}
    \item \texttt{set\_year($DateTime$,$Year$)}
    \item \texttt{time\_string($DateTime$) = $String$}
    \item \texttt{year($DateTime$) = $Year$}
\end{itemize}
\end{scriptsize}
}
\end{multicols}
\ignore{
\end{document}
}

\end{adjustwidth}
%\cleardoublepage
\clearpage
\phantomsection
\addcontentsline{toc}{chapter}{Index}
\printindex
\end{document}
