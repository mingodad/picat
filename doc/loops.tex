\ignore{
\documentstyle[11pt]{report}
\textwidth 13.7cm
\textheight 21.5cm
\newcommand{\myimp}{\verb+ :- +}
\newcommand{\ignore}[1]{}
\def\definitionname{Definition}

\makeindex
\begin{document}

}
\chapter{Assignments and Loops\label{chapter:loops}}
This chapter discusses variable assignments\index{assignment}, loop constructs, and list and array comprehensions\index{list comprehension} in Picat.  It describes the \emph{scope}\index{scope} of an assigned\index{assignment} variable, indicating where the variable is defined, and where it is not defined.  Finally, it shows how assignments\index{assignment}, loops, and list comprehensions\index{list comprehension} are related, and how they are compiled.  

\section{Assignments}
Picat variables are \emph{single-assignment}\index{assignment}\index{single-assignment}, meaning that once a variable is bound to a value, the variable cannot be bound again.  In order to simulate imperative\index{imperative} language variables, Picat provides the assignment\index{assignment} operator \verb+:=+.  An assignment\index{assignment} takes the form $LHS $$:$$=$$ RHS$, where $LHS$ is either a variable or an access of a compound value\index{compound value} in the form \texttt{$X$[\ldots]}. When $LHS$ is an access in the form $X[I]$, the component of $X$ indexed $I$ is updated.  This update is undone if execution backtracks over this assignment\index{assignment}.

\subsection*{Example}
\begin{verbatim}
test => X = 0, X := X + 1,  X := X + 2, write(X).
\end{verbatim}

The compiler needs to give special consideration to the \emph{scope}\index{scope} of a variable.  The scope\index{scope} of a variable refers to the parts of a program where a variable occurs.  

Consider the \texttt{test} example.  This example binds \texttt{X} to $0$.  Then, the example tries to bind \texttt{X} to \texttt{X + 1}.  However, \texttt{X} is still in scope\index{scope}, meaning that \texttt{X} is already bound to $0$.  Since \texttt{X} cannot be bound again, the compiler must perform extra operations in order to manage assignments\index{assignment} that use the \texttt{:=} operator. 

In order to handle assignments\index{assignment}, Picat creates new variables at compile time.  In the \texttt{test} example, at compile time, Picat creates a new variable, say \texttt{X1}, to hold the value of \texttt{X} after the assignment\index{assignment} \verb-X := X + 1-. Picat replaces \texttt{X} by \texttt{X1} on the LHS of the assignment\index{assignment}.  All occurrences of \texttt{X} after the assignment\index{assignment} are replaced by \texttt{X1}.  When encountering \verb-X1 := X1 + 2-, Picat creates another new variable, say \texttt{X2}, to hold the value of \texttt{X1} after the assignment\index{assignment}, and replaces the remaining occurrences of \texttt{X1} by \texttt{X2}. When \texttt{write(X2)}\index{\texttt{write/1}} is executed, the value held in \texttt{X2}, which is 3, is printed.  This means that the compiler rewrites the above example as follows:
\begin{verbatim}
test => X = 0, X1 = X + 1, X2 = X1 + 2, write(X2).
\end{verbatim}

\subsection{If-Else}
This leads to the question: what does the compiler do if the code branches?  Consider the following code skeleton.
\subsection*{Example}
\begin{verbatim}
if_ex(Z) =>
    X = 1, Y = 2,
    if Z > 0 then
       X := X * Z    
    else
       Y := Y + Z
    end,
    println([X,Y]).
\end{verbatim}

The \texttt{if\_ex} example performs exactly one assignment\index{assignment}.  At compilation time, the compiler does not know whether or not \texttt{ Z$>$0} evaluates to \texttt{true}\index{\texttt{true}}.  Therefore, the compiler does not know whether to introduce a new variable for \texttt{X} or for \texttt{Y}.

Therefore, when an if-else statement\index{if statement} contains an assignment\index{assignment}, the compiler rewrites the if-else statement\index{if statement} as a predicate.  For example, the compiler rewrites the above example as follows:

\begin{verbatim}
if_ex(Z) => 
   X = 1, Y = 2, 
   p(X, Xout, Y, Yout, Z),
   println([Xout,Yout]).

p(Xin, Xout, Yin, Yout, Z), Z > 0 =>
   Xout = X * Z,
   Yout = Yin.
p(Xin, Xout, Yin, Yout) =>
   Xout = Xin,
   Yout = Y + Z.
\end{verbatim}
One rule is generated for each branch of the if-else statement\index{if statement}.  For each variable \texttt{V} that occurs on the LHS of an assignment\index{assignment} statement that is inside of the if-else statement\index{if statement}, predicate \texttt{p} is passed two arguments, \texttt{Vin} and \texttt{Vout}.  In the above example, \texttt{X} and \texttt{Y} each occur on the LHS of an assignment\index{assignment} statement.  Therefore, predicate \texttt{p} is passed the parameters \texttt{Xin}, \texttt{Xout}, \texttt{Yin}, and \texttt{Yout}. 

\section{Types of Loops}
Picat has three types of loop statements for programming repetitions: \texttt{foreach}\index{foreach loop}, \texttt{while}\index{while loop}, and \texttt{do-while}\index{do-while loop}.

\subsection{Foreach Loops}
A \texttt{foreach} loop\index{foreach loop} has the form:
\begin{tabbing}
aa \= aaa \= aaa \= aaa \= aaa \= aaa \= aaa \kill
\> \texttt{foreach ($E_1$ in $D_1$, $Cond_1$, $\ldots$, $E_n$ in $D_n$, $Cond_n$)}  \\
\> \> $Goal$ \\
\>  \texttt{end} 
\end{tabbing}
Each $E_i$ is an \emph{iterating pattern}\index{iterator}.  Each $D_i$ is an expression that gives a \emph{compound value}\index{compound value}.  Each $Cond_i$ is an optional \emph{condition} on iterators\index{iterator} $E_1$ through $E_i$.

Foreach loops\index{foreach loop} can be used to iterate through compound values\index{compound value}, as in the following examples.

\subsection*{Example}
\begin{verbatim}
loop_ex1 =>
    L = [17, 3, 41, 25, 8, 1, 6, 40],
    foreach (E in L)
        println(E)
    end.

loop_ex2(Map) =>
    foreach(Key=Value in Map)
        writef("%w=%w\n", Key, Value)
    end.
\end{verbatim}

The \texttt{loop\_ex1} example iterates through a list.  The \texttt{loop\_ex2} example iterates through a map, where \texttt{Key=Value} is the iterating pattern\index{iterator}.

The \texttt{loop\_ex1} example can also be written, using a \emph{failure-driven loop}\index{failure-driven loop}, as follows.

\subsection*{Example}
\begin{verbatim}
loop_ex1 =>
    L = [17, 3, 41, 25, 8, 1, 6, 40],
    (    member(E, L),
         println(E),
         fail
    ;
         true
    ).
\end{verbatim}


Recall that the range $Start .. Step .. End$ stands for a list of numbers.  Ranges can be used as compound values\index{compound value} in iterators\index{iterator}.
\subsection*{Example}
\begin{verbatim}
loop_ex3 =>
    foreach(E in 1 .. 2 .. 9)
        println(E)
    end.
\end{verbatim}

Also recall that the function \texttt{zip($List_1$, $List_2$, $\ldots$, $List_n$)}\index{\texttt{zip}} returns a list of tuples. This function can be used to simultaneously iterate over multiple lists.

\subsection*{Example:}
\begin{verbatim}
loop_ex_parallel =>
    foreach(Pair in zip(1..2, [a,b]))
        println(Pair)
    end.
\end{verbatim}


\subsection{Foreach Loops with Multiple Iterators}
Each of the previous examples uses a single iterator\index{iterator}.  Foreach loops\index{foreach loop} can also contain multiple iterators\index{iterator}.
\subsection*{Example:}
\begin{verbatim}
loop_ex4 =>
    L = [2, 3, 5, 10],
    foreach(I in L, J in 1 .. 10, J mod I != 0)
        printf("%d is not a multiple of %d%n", J, I)
    end.
\end{verbatim}

If a foreach loop\index{foreach loop} has multiple iterators\index{iterator}, then it is compiled into a series of nested foreach loops\index{foreach loop}\index{nested loop} in which each nested loop\index{nested loop} has a single iterator\index{iterator}.  In other words, a foreach loop\index{foreach loop} with multiple iterators\index{iterator} executes its goal once for every possible combination of values in the iterators\index{iterator}.

The foreach loop\index{foreach loop} in \texttt{loop\_ex4} is the same as the nested loop\index{nested loop}:
\begin{verbatim}
loop_ex5 =>
    L = [2, 3, 5, 10],
    foreach(I in L)
        foreach(J in 1..10)
            if J mod I != 0 then
               printf("%d is not a multiple of %d%n", J, I)
            end
        end
    end.
\end{verbatim}

\subsection{While Loops}
A \texttt{while} loop\index{while loop} has the form:
\begin{tabbing}
aa \= aaa \= aaa \= aaa \= aaa \= aaa \= aaa \kill
\> \texttt{while ($Cond$)} \\
\> \> $Goal$  \\
\>  \texttt{end}
\end{tabbing} 
As long as $Cond$ succeeds, the loop will repeatedly execute $Goal$.  
\subsection*{Example:}
\begin{verbatim}
loop_ex6 =>
    I = 1, 
    while (I <= 9)
        println(I),
        I := I + 2
    end.

loop_ex7 =>
    J = 6,
    while (J <= 5)
        println(J),
        J := J + 1
    end.

loop_ex8 =>
    E = read_int(),
    while (E mod 2 == 0; E mod 5 == 0)
        println(E),
        E := read_int()
    end.

loop_ex9 =>
    E = read_int(),
    while (E mod 2 == 0, E mod 5 == 0)
        println(E),
        E := read_int()
    end.
\end{verbatim}

The while loop\index{while loop} in \texttt{loop\_ex6} prints all of the odd numbers between $1$ and $9$.  It is similar to the foreach loop\index{foreach loop}
\begin{verbatim}
foreach(I in 1 .. 2 .. 9)
    println(I)
end.
\end{verbatim}

The while loop\index{while loop} in \texttt{loop\_ex7} never executes its goal.  \texttt{J} begins at $6$, so the condition \texttt{J <= 5} is never true, meaning that the body of the loop does not execute.

The while loop\index{while loop} in \texttt{loop\_ex8} demonstrates a compound condition.  The loop executes as long as the value that is read into \texttt{E} is either a multiple of $2$ or a multiple of $5$.

The while loop\index{while loop} in \texttt{loop\_ex9} also demonstrates a compound condition.  Unlike in \texttt{loop\_ex8}, in which either condition must be true, in \texttt{loop\_ex9}, both conditions must be true.  The loop executes as long as the value that is read into \texttt{E} is both a multiple of $2$ and a multiple of $5$. 

\subsection{Do-while Loops}
A \texttt{do-while} loop\index{do-while loop} has the form:
\begin{tabbing}
aa \= aaa \= aaa \= aaa \= aaa \= aaa \= aaa \kill
\> \texttt{do} \\
\> \> $Goal$  \\
\>  \texttt{while ($Cond$)}
\end{tabbing} 
A do-while loop\index{do-while loop} is similar to a while loop\index{while loop}, except that a do-while loop\index{do-while loop} executes $Goal$ one time before testing $Cond$.  The following example demonstrates the similarities and differences between do-while loops\index{do-while loop} and while loops\index{while loop}.
\subsection*{Example}
\begin{verbatim}
loop_ex10 =>
    J = 6,
    do
        println(J),
        J := J + 1
    while (J <= 5).
\end{verbatim}

Unlike \texttt{loop\_ex7}, \texttt{loop\_ex10} executes its body once.  Although \texttt{J} begins at $6$, the do-while loop\index{do-while loop} prints \texttt{J}, and increments \texttt{J} before evaluating the condition \texttt{J <= 5}.

\section{List and Array Comprehensions}
A \emph{list comprehension}\index{list comprehension} is a special functional notation for creating lists.  List comprehensions\index{list comprehension} have a similar format to foreach loops\index{foreach loop}.
\begin{tabbing}
aa \= aaa \= aaa \= aaa \= aaa \= aaa \= aaa \kill
\> \> \texttt{[$T$ : $E_1$ \texttt{in} $D_1$, $Cond_1$, $\ldots$, $E_n$ in $D_n$, $Cond_n$]} 
\end{tabbing}
$T$ is an expression.  Each $E_i$ is an \emph{iterating pattern}\index{iterator}.  Each $D_i$ is an expression that gives a \emph{compound value}\index{compound value}.  Each $Cond_i$ is an optional \emph{condition} on iterators\index{iterator} $E_1$ through $E_i$.

An array comprehension\index{array comprehension} takes the following form:
\begin{tabbing}
aa \= aaa \= aaa \= aaa \= aaa \= aaa \= aaa \kill
\> \> \texttt{\{$T$ : $E_1$ \texttt{in} $D_1$, $Cond_1$, $\ldots$, $E_n$ in $D_n$, $Cond_n$\}} 
\end{tabbing}
It is the same as:
\begin{tabbing}
aa \= aaa \= aaa \= aaa \= aaa \= aaa \= aaa \kill
\> \> \texttt{to\_array([$T$ : $E_1$ \texttt{in} $D_1$, $Cond_1$, $\ldots$, $E_n$ in $D_n$, $Cond_n$])} 
\end{tabbing}

\subsection*{Example}
\begin{verbatim}
picat> L = [(A, I) : A in [a, b], I in 1 .. 2].
L = [(a , 1),(a , 2),(b , 1),(b , 2)]

picat> L = {(A, I) : A in [a, b], I in 1 .. 2}.
L = {(a , 1),(a , 2),(b , 1),(b , 2)}
\end{verbatim}

\section{Compilation of Loops}
Variables that occur in a loop, but do not occur before the loop in the outer scope\index{scope}, are local\index{local variable} to each iteration of the loop. For example, in the rule
\begin{verbatim}
p(A) =>
    foreach (I in 1 .. A.length)
        E = A[I],
        println(E)
    end.
\end{verbatim}
the variables \texttt{I} and \texttt{E} are local\index{local variable}, and each iteration of the loop has its own values for these variables.

Consider the example:
\subsection*{Example}
\begin{verbatim}
while_test(N) =>
    I = 1,
    while (I <= N)
        I := I + 1,
        println(I)
    end.
\end{verbatim}
In this example, the while loop\index{while loop} contains an assignment\index{assignment} statement.  As mentioned above, at compilation time, Picat creates new variables in order to handle assignments\index{assignment}.  One new variable is created for each assignment\index{assignment}.  However, when this example is compiled, the compiler does not know the number of times that the body of the while loop\index{while loop} can be executed.  This means that the compiler does not know how many times the assignment\index{assignment} \texttt{I := I + 1} will occur, and the compiler is unable to create new variables for this assignment\index{assignment}.  In order to solve this problem, the compiler compiles while loops\index{while loop} into tail-recursive\index{tail recursion} predicates.

In the \texttt{while\_test} example, the while loop\index{while loop} is compiled into:
\begin{verbatim}
while_test(N) =>
    I = 1,
    p(I, N).

p(I, N), I <= N => 
    I1 = I + 1,
    println(I1),
    p(I1, N).
p(_, _) => true. 
\end{verbatim}

Note that the first rule of the predicate \texttt{p(I, N)} has the same condition as the while loop\index{while loop}.  The second rule, which has no condition, terminates the while loop\index{while loop}, because the second rule is only executed if \texttt{I > N}.  The call \texttt{p(I1, N)} is the tail-recursive\index{tail recursion} call, with \texttt{I1} storing the modified value.

Suppose that a while loop\index{while loop} modifies a variable that is then used outside of the while loop\index{while loop}.   For each modified variable \texttt{V} that is used after the while loop\index{while loop}, predicate \texttt{p} is passed two arguments, \texttt{Vin} and \texttt{Vout}.  Then, a predicate that has the body \texttt{true}\index{\texttt{true}} is not sufficient to terminate the compiled while loop\index{while loop}.  Instead, a predicate fact must be used, as in the next example.

The next example demonstrates a loop that has multiple accumulators\index{accumulator}, and that modifies values which are then used outside of the loop.
\subsection*{Example}
\begin{verbatim}
min_max([H|T], Min, Max) =>
    LMin = H,
    LMax = H,
    foreach (E in T)
        LMin := min(LMin, E),
        LMax := max(LMax, E)
    end,
    Min = LMin,
    Max = LMax.
\end{verbatim}

This loop finds the minimum and maximum values of a list.  The loop is compiled to:
\begin{verbatim}
min_max([H|T], Min, Max) =>
    LMin = H,
    LMax = H,
    p(T, LMin, LMin1, LMax, LMax1),
    Min = LMin1,
    Max = LMax1.

p([], MinIn, MinOut, MaxIn, MaxOut) =>
    MinOut = MinIn,
    MaxOut= MaxIn.
p([E|T], MinIn, MinOut, MaxIn, MaxOut) =>
    Min1 = min(MinIn, E),
    Max1 = max(MaxIn, E),
    p(T, Min1, MinOut, Max1, MaxOut).
\end{verbatim}
Notice that there are multiple accumulators\index{accumulator}: \texttt{MinIn} and \texttt{MaxIn}.  Since the \texttt{min\_max} predicate returns two values, the accumulators each have an \textit{in} variable (\texttt{MinIn} and \texttt{Maxin}) and an \textit{out} variable (\texttt{MinOut} and \texttt{MaxOut}).   If the first parameter of predicate \texttt{p} is an empty list, then \texttt{MinOut} is set to the value of \texttt{MinIn}, and \texttt{MaxOut} is set to the value of \texttt{MaxIn}.

Foreach\index{foreach loop} and do-while loops\index{do-while loop} are compiled in a similar manner to while loops\index{while loop}.

\subsubsection{Nested Loops}
As mentioned above, variables that only occur within a loop are local\index{local variable} to each iteration of the loop.  In nested loops\index{nested loop}, variables that are local\index{local variable} to the outer loop are global to the inner loop.  In other words, if a variable occurs in the outer loop, then the variable is also visible in the inner loop.  However, variables that are local\index{local variable} to the inner loop are not visiable to the outer loop.

For example, consider the nested loops\index{nested loop}:
\begin{verbatim}
nested =>
    foreach (I in 1 .. 10)
        printf("Numbers between %d and %d ", I, I * I),
        foreach (J in I .. I * I)
            printf("%d ", J)
        end,
        nl
    end.
\end{verbatim}
Variable \texttt{I} is local\index{local variable} to the outer foreach loop\index{foreach loop}, and is global to the inner foreach loop\index{foreach loop}.  Therefore, iterator\index{iterator} \texttt{J} is able to iterate from \texttt{I} to \texttt{I * I} in the inner foreach loop\index{foreach loop}.  Iterator\index{iterator} \texttt{J} is local\index{local variable} to the inner loop, and does not occur in the outer loop.

Since a foreach loop\index{foreach loop} with \texttt{N} iterators\index{iterator} is converted into \texttt{N} nested foreach loops\index{foreach loop}\index{nested loop}, the order of the iterators\index{iterator} matters.  

\subsection{List Comprehensions}
List comprehensions\index{list comprehension} are compiled into foreach loops\index{foreach loop}.

\subsection*{Example}
\begin{verbatim}
comp_ex =>
    L = [(A, X) : A in [a, b], X in 1 .. 2].
\end{verbatim}
This list comprehension\index{list comprehension} is compiled to:
\begin{verbatim}
comp_ex =>
    List = L, 
    foreach (A in [a, b], X in 1 .. 2)
        L = [(A, X) | T], 
        L := T
    end,
    L = [].
\end{verbatim}

\subsection*{Example}
\begin{verbatim}
make_list1 =>
    L = [Y : X in 1..5],
    write(L).

make_list2 =>
    Y = Y,
    L = [Y : X in 1..5],
    write(L).
\end{verbatim}
Suppose that a user would like to create a list \texttt{[Y, Y, Y, Y, Y]}.  The \texttt{make\_list1} predicate incorrectly attempts to make this list; instead, it outputs a list of 5 different variables since \texttt{Y} is local.  In order to make all five variables the same, \texttt{make\_list2} makes variable \texttt{Y} global, by adding the line \texttt{Y = Y} to globalize \texttt{Y}.

\ignore{
\end{document}
}
